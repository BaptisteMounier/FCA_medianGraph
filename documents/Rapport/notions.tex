\chapter{Notions nécessaires}

\section{Analyse de Concepts Formels}

L'Analyse Concepts Formels utilise des matrices binaires mettant en relation des objets avec des attributs, une matrice porte le nom de contexte et est défini par $C(O, A, I)$ où $O$ est l'ensemble des objets, $A$ l'ensemble des attributs et $I$ l'ensemble des relations entre les objets et les attributs.

\begin{definition}[Contexte]
Soit un contexte $C(O, A, I)$ dans lequel $O$ est l'ensemble des objets, $A$ l'ensemble des attributs et $I$ l'ensemble des relations entre les objets et les attributs.
\end{definition}

Nous définitions une opération, \guillemotleft{} prime \guillemotright{} pour les objets et attributs, notée \guillemotleft{} ' \guillemotright{}. Elle permet de passer des objets aux attributs et inversement. On lui donne un objet (resp. attribut) ou un ensemble d'objets (resp. attributs) et elle nous renvois l'ensemble des attributs (resp. objets) qui sont en correspondance.

\begin{definition}[Connexion de Gallois]
Pour $X \subseteq O$ et $Y \subseteq A$, on définie :
\begin{itemize}
	\item $X' = \{y \in A : (x, y) \in I$, $\forall x \in X\}$
	\item $Y' = \{x \in O : (x, y) \in I$, $\forall y \in Y\}$
\end{itemize}
\end{definition}

Dans la figure \ref{def_contexte}, nous avons $O = \{chat, chien, aigle, mouette, poule\}$, $A = \{griffes, ailes, nyclatope, vol\}$ et $I = \{(chat, griffes), (chat, nyctalope), (chien, griffes), (aigle, griffes), ...\}$. Nous avons entre autres pour les objets $chat' = \{griffes, nyctalope\}$ et $\{chat, aigle\}' = \{griffes\}$ et inversement pour les attributs $griffes' = \{chat, chien, aigle, poule\}$ et $\{griffes, ailes\}' = \{aigle, poule\}$.

\begin{figure}[H]
	\begin{center}
		\begin{tabular}{ l | c c c c }
			 & $griffes$ & $ailes$ & $nyctalope$ & $vol$ \\
			\hline
			$chat$ & x & & x & \\
			$chien$ & x & & & \\
			$aigle$ & x & x & & x \\
			$mouette$ & & x & & x \\
			$poule$ & x & x & & \\
		\end{tabular}
	\end{center}
	\caption{Contexte}
	\label{def_contexte}
\end{figure} 

Nous pouvont effectuer plusieurs opérations sur les contextes sans réelles pertes d'informations sur la structure. La première est la clarification, cela consiste à ne garder que les lignes et les colonnes qui ne sont pas en doublon.

\begin{definition}[Contexte clarifié]
Soit un contexte clarifié $C(O, A, I)$ :
\begin{itemize}
	\item $\forall x1, x2 \in O$ si ${x1}' = {x2}'$ alors $x1 = x2$
	\item $\forall y1, y2 \in A$ si ${y1}' = {y2}'$ alors $y1 = y2$
\end{itemize}
\end{definition}

À partir du contexte non clarifié de la figure \ref{def_contexte_non_clarifie} nous obtenu le contexte clarifié de la figure \ref{def_contexte_clarifie} avec la suppression de la ligne $5$ qui est le doublon de la ligne $2$ et la colonne $e$ qui est le doublon de la $b$. Nous pouvons le voir à travers les opérations de prime, deux objets (resp. attributs) sont en doublon losqu'ils obtiennent le même résultat. Nous obtenons $2' = \{a\}$ et $5' = \{a\}$. Il est important de bien comprendre que malgré la suppression de lignes ou de colonnes dans cette opération, nous ne perdons aucune données.

\begin{figure}[H]
	\begin{minipage}[c]{0.5\textwidth}
	\begin{center}
		\begin{tabular}{ l | c c c c c }
			 & $a$ & $b$ & $c$ & $d$ & $e$ \\
			\hline
			$1$ & x & & & x & \\
			$2$ & x & & & & \\
			$3$ & x & x & x & & x \\
			$4$ & & x & x & & x \\
			$5$ & x & & & & \\
			$6$ & x & & x & & \\
		\end{tabular}
	\end{center}
	\caption{Contexte non clarifié}
	\label{def_contexte_non_clarifie}
	\end{minipage}
	\begin{minipage}[c]{0.5\textwidth}
	\begin{center}
		\begin{tabular}{ l | c c c c }
			 & $a$ & $b$ & $c$ & $d$ \\
			\hline
			$1$ & x & & & x \\
			$2$ & x & & & \\
			$3$ & x & x & x & \\
			$4$ & & x & x & \\
			$6$ & x & & x & \\
		\end{tabular}
	\end{center}
	\caption{Contexte clarifié}
	\label{def_contexte_clarifie}
	\end{minipage}
\end{figure}

La seconde opération est la plus utilisée et va plus loin dans la réduction de la taille du contexte sans perte de données, les contextes ainsi obtenus sont dit \guillemotleft{} contexte réduit \guillemotright \footnote{\guillemotleft{} standard context \guillemotright{} pour la documentation anglaise}. Le principe est toujours de supprimer les lignes et les colonnes dont on peut se passer, celles qui sont recalculables à partir de celles qu'on garde. On ne garde que les lignes et les colonnes qui ne sont pas l'intersection d'une ou plusieurs autres.

\begin{definition}[Contexte réduit]
Un contexte clarifié est réduit $C(J, M, I)$ si et seulement si :
\begin{itemize}
	\item $\forall x \in C, \forall X \subseteq C$, si $x' = X'$ alors $x \in X$
	\item $\forall x \in M, \forall X \subseteq M$, si $x' = X'$ alors $x \in X$
\end{itemize}
\end{definition}

Dans la figure \ref{def_contexte_non_reduit} la ligne $2' = \{a\}$ est l'intersection des lignes $1' = \{a, d\}$ et $3' = \{a, b, c\}$ ou $1' = \{a, d\}$ et $6' = \{a, c\}$, on peut donc la supprimer. En revanche, nous n'avons aucune colonne dans ce cas ici. Lorsqu'on effectue la réduction d'un contexte, il est systématiquement sous entendu que l'opération de clarification est également effectuée.

\begin{figure}[H]
	\begin{minipage}[c]{0.5\textwidth}
	\begin{center}
		\begin{tabular}{ l | c c c c }
			 & $a$ & $b$ & $c$ & $d$ \\
			\hline
			$1$ & x & & & x \\
			$2$ & x & & & \\
			$3$ & x & x & x & \\
			$4$ & & x & x & \\
			$6$ & x & & x & \\
		\end{tabular}
	\end{center}
	\caption{Contexte non réduit}
	\label{def_contexte_non_reduit}
	\end{minipage}
	\begin{minipage}[c]{0.5\textwidth}
	\begin{center}
		\begin{tabular}{ l | c c c c }
			 & $a$ & $b$ & $c$ & $d$ \\
			\hline
			$1$ & x & & & x \\
			$3$ & x & x & x & \\
			$4$ & & x & x & \\
			$6$ & x & & x & \\
		\end{tabular}
	\end{center}
	\caption{Contexte réduit}
	\label{def_contexte_reduit}
	\end{minipage}
\end{figure}

Nous pouvons maintenant définir la notion de concept. Cela correspond aux rectangles maximaux qui se trouvent dans le contexte, regroupant tous les objets qui font partis du concept et tous les attributs qu'ils ont en commun. Un concept est l'ensemble des objets et attributs tel que tous les objets du concepts partagent les attributs du concepts et qu'il n'existe pas d'autre objet partageant les attributs du concept et pas d'autre attributs qui sont partagés par tous les objets du concept.

\begin{definition}[Concept]
Soit un concept $c$, $X \subseteq O$ et $Y \subseteq A$ :
\begin{itemize}
	\item $c = \{X, Y\}$
	\item $\forall x \in O$, $x \in X \Leftrightarrow x' \subseteq Y$
	\item $\forall y \in A$, $y \in Y \Leftrightarrow y' \subseteq X$
\end{itemize}
\end{definition}

Sur le contexte de la figure \ref{def_contexte_reduit} nous avons entre autre le concept contenant les objets 3 et 4 et les attributs $b$ et $c$ dont les correspondances forment un rectangle plein.

\bigbreak

Maintenant que nous avons défini la notion de concept, nous pouvons définir les treillis de concepts, également appelés treillis de Galois. Ce treillis se base sur une relation d'ordre entre les concepts exprimée sur un diagramme de Hasse. C'est un diagramme orientés du bas vers le  haut où chaque élément, ici il s'agit du concept, est un n\oe ud et chaque relation d'ordre est un arc.

\begin{definition}[Treillis de concepts]
Soit deux concepts $c_1 = (O_1, A_1)$ et $c_2 = (O_2, A_2)$, on a $c_1 \leq c_2$ si et seulement si $O_1 \subseteq O_2$ ou $A_1 \supseteq A_2$.
\end{definition}

En plus des propriétés précédentes, un contexte peut contenir ce qu'on appelle des relations fléches qui ne viennent pas ajouter de l'information mais permettre une extraction plus rapide à porté humaine. Il en existe de trois types $\uparrow$, $\downarrow$, et $\updownarrow$.

\begin{definition}[Relations flèches]
Soit un contexte $C(O, A, I)$, $o \in O$, $a \in A$ :
\begin{itemize}
	\item $o \uparrow a$ ssi $(o, a) \not \in I$ et si $\exists x \in A$, $a' \subset x'$, $(o, x) \in I$ : $\forall y \in A$, $a' \subset y' \Rightarrow y = x$ 
	\item $o \downarrow a$ ssi $(o, a) \not \in I$ et si $\exists x \in O$, $o' \subset x'$, $(x, a) \in I$ : $\forall y \in O$, $o' \subset y' \Rightarrow y = x$ 
	\item $o \updownarrow a$ ssi $o \uparrow a$ et $o	 \downarrow a$.
\end{itemize}
\end{definition}

\begin{figure}[H]
	\begin{minipage}{0.4\textwidth}
	\begin{center}
		\begin{tabular}{ l | c c c c c }
			 & A & B & C & D & E \\
			\hline
			1 & x & x & x & x & $\updownarrow$ \\
			2 & x & $\updownarrow$ & x & x & x \\
			3 & x & x & x & $\updownarrow$ & \\
			4 & x & $\uparrow$ & $\updownarrow$ & x & x \\
			5 & x & x & $\updownarrow$ & $\uparrow$ & \\
			6 & $\updownarrow$ & x & & & \\
		\end{tabular}
		\caption{Contexte avec relations fléches}
	\end{center}
	\end{minipage}
	\begin{minipage}{0.8\textwidth}
		\begin{center}
			\begin{tikzpicture}
				\node [wnode, label=below:{$\bot$}] (bot) at (0,0) {};
				\node [wnode, label=left:{1}] (1) at (-1,1) {};
				\node [wnode, label=right:{2}] (2) at (1,1) {};
				\node [wnode] (1_2) at (0,2) {};
				\node [wnode, label=left:{3}] (3) at (-2,2) {};
				\node [wnode, label=right:{4, E}] (4) at (2,2) {};
				\node [wnode, label=left:{5}] (5) at (-2,3) {};
				\node [wnode, label=right:{C}] (C) at (0,3) {};
				\node [wnode, label=right:{D}] (D) at (2,3) {};
				\node [wnode, label=left:{6, B}] (6) at (-2,4) {};
				\node [wnode, label=right:{A}] (A) at (0,4) {};
				\node [wnode, label={$\top$}] (top) at (0,5) {};
				
				\path [line] (bot) -- (1);
				\path [line] (bot) -- (2);
				\path [line] (1) -- (3);
				\path [line] (1) -- (1_2);
				\path [line] (2) -- (4);
				\path [line] (2) -- (1_2);
				\path [line] (3) -- (5);
				\path [line] (3) -- (C);
				\path [line] (1_2) -- (C);
				\path [line] (1_2) -- (D);
				\path [line] (4) -- (D);
				\path [line] (5) -- (6);
				\path [line] (5) -- (A);
				\path [line] (C) -- (A);
				\path [line] (D) -- (A);
				\path [line] (6) -- (top);
				\path [line] (A) -- (top);
			\end{tikzpicture}
		\end{center}
		\caption{Treillis associé}
	\end{minipage}
\end{figure}

\section{Ensemble ordonné}

Un ensemble ordonné est un ensemble $P$ d'éléments avec une relation d'ordre $\leq$, qu'on note $(P, \leq)$. Pour le représenter, on utilise des diagrammes de Hasse. Pour un ensemble $X \subseteq P$ on note $\uparrow \! X$ le filtre (resp. $\downarrow \! X$ l'idéal) de $X$. Pour un élément $x \in P$, on note $\uparrow \! x$ le filtre principal (resp. $\downarrow \! x$ l'idéal principal) de $x$ tel que pour $\uparrow \! x = {y}$ (resp. $\downarrow \! x = {y}$) on a $x \leq y$ (resp. $y \leq x$).

\begin{definition}[Ensemble ordonné]
Soit un ensemble ordonné $(P, \leq)$, $X \subseteq P$ et $Y \subseteq P$ :
\begin{itemize}
	\item $\forall x, y \in P$, on a $x \leq y$ ou $y \leq x$
	\item $\uparrow \! X = Y \Rightarrow x \leq y$ $\forall x \in X, \forall y \in Y$
	\item $\downarrow \! X = Y \Rightarrow y \leq x$ $\forall x \in X, \forall y \in Y$
\end{itemize}
\end{definition}

Dans la figure \ref{hasse_filtres_ideaux} nous avons $\uparrow \! a = {a, b, c}$ et $\downarrow \! b = {a, b}$.

\begin{figure}[H]
	\begin{minipage}{0.3\textwidth}
	\begin{center}
		\begin{tikzpicture}
			\node [wnode, label=below:{$a$}] (a) at (-1, 0) {};
			\node [wnode, label=left:{$b$}] (b) at (-2, 1) {};
			\node [wnode, label=right:{$c$}] (c) at (0, 1) {};
			
			\path [line] (a) -- (b);
			\path [line] (a) -- (c);
		\end{tikzpicture}
	\end{center}
	\end{minipage}
	\begin{minipage}{0.3\textwidth}
	\begin{center}
		\begin{tikzpicture}
			\node [bnode, label=below:{$\uparrow \! a = \{a, b, c\}$}] (a) at (-1, 0) {};
			\node [bnode, label=left:{$b$}] (b) at (-2, 1) {};
			\node [bnode, label=right:{$c$}] (c) at (0, 1) {};
			
			\path [line] (a) -- (b);
			\path [line] (a) -- (c);
			\draw [very thick, dotted] plot [smooth, tension=0.5] coordinates {(-2.2, 1) (-1, -0.2) (0.2, 1)};
		\end{tikzpicture}
	\end{center}
	\end{minipage}
	\begin{minipage}{0.3\textwidth}
	\begin{center}
		\begin{tikzpicture}
			\node [bnode, label=below:{$a$}] (a) at (-1, 0) {};
			\node [bnode, label={$\downarrow \! b = \{a, b\}$}] (b) at (-2, 1) {};
			\node [wnode, label=right:{$c$}] (c) at (0, 1) {};
			
			\path [line] (a) -- (b);
			\path [line] (a) -- (c);
			\draw [very thick, dotted] plot [smooth, tension=0.5] coordinates {(-2.2, 0) (-2, 1.2) (-0.8, 0)};
		\end{tikzpicture}
	\end{center}
	\end{minipage}
	\caption{Diagrammes de Hasse avec filtres et idéaux}
	\label{hasse_filtres_ideaux}
\end{figure}

\section{Treillis}

Un treillis est un ensemble ordonné $(T, \leq)$ sur lequel on définit pour chaque tuple un supremum ou borne supérieure noté $\vee$ et un infimum ou borne inférieure noté $\wedge$, on le note $(T, \leq, \vee, \wedge)$.\todo{Comment définir les notions de borne sup et borne inf ?} Pour trois éléments $x, y, z \in P$ tel que $z \leq x$ et $z \leq y$ on peut dire que $x \vee y = z$. Le second est celui d'infimum ou borne inférieure noté $\wedge$. Et de la même façon pour trois éléments $x, y, z \in P$ tel que $x \leq z$ et $y \leq z$ on peut dire que $x \wedge y = z$. Lorsqu'un élément est la borne supérieure (resp. inférieure) de tous les autres éléments on dit qu'il ferme le treillis par les supremums (resp. infimums), on note cet élément $\top$ (resp.$\bot$).

\begin{definition}[Treillis]
Soit un treillis $(T, \leq, \vee, \wedge)$ :
\begin{itemize}
	\item $\forall x, y \in T$, $\exists z \in T : x \vee y = z$
	\item $\forall x, y \in T$, $\exists z \in T : x \wedge y = z$
\end{itemize}
\end{definition}

Sur notre exemple en figure \ref{treillis_sup_inf} nous avons $a = b \vee c$, $d = b \wedge c$, $\top = d$ et $\bot = a$.

\begin{figure}[H]
	\begin{minipage}{0.5\textwidth}
	\begin{center}
		\begin{tikzpicture}
			\node [bnode, label=below:{$a = b \vee c$}] (a) at (0, 0) {};
			\node [bnode, label=left:{$b$}] (b) at (-1, 1) {};
			\node [bnode, label=right:{$c$}] (c) at (1, 1) {};
			\node [wnode, label={$d$}] (d) at (0, 2) {};
			
			\path [line] (a) -- (b);
			\path [line] (a) -- (c);
			\path [line] (b) -- (d);
			\path [line] (c) -- (d);
		\end{tikzpicture}
	\end{center}
	\end{minipage}
	\begin{minipage}{0.5\textwidth}
	\begin{center}
		\begin{tikzpicture}
			\node [wnode, label=below:{$a$}] (a) at (0, 0) {};
			\node [bnode, label=left:{$b$}] (b) at (-1, 1) {};
			\node [bnode, label=right:{$c$}] (c) at (1, 1) {};
			\node [bnode, label={$d = b \wedge c$}] (d) at (0, 2) {};
			
			\path [line] (a) -- (b);
			\path [line] (a) -- (c);
			\path [line] (b) -- (d);
			\path [line] (c) -- (d);
		\end{tikzpicture}
	\end{center}
	\end{minipage}
	\caption{Treillis avec supremum et infimum}
	\label{treillis_sup_inf}
\end{figure}

Un élément qui est pas le supremum (resp. infimum) d'autres éléments est un $\vee$-irréductible (resp. $\wedge$-irréductible) et on note l'ensemble de ces élément $J(T)$ (resp. $M(T)$). Bien que le $\top$ et le $\bot$ soient des irréductibles, il est commun de ne pas toujours les prendre en considération, nous ne les prendront pas non plus dans ce rapport, il sera précisé lorsqu'ils le seront.

\begin{definition}[$\vee$-irreductible et $\wedge$-irreductible]
Soit un treillis $(T, \leq, \vee, \wedge)$, $x, y, z \in T$ :
\begin{itemize}
	\item $x$ est $\vee$-irreductible ssi $x \vee y = z \Rightarrow x = z$
	\item $x$ est $\wedge$-irreductible ssi $x \wedge y = z \Rightarrow x = z$
\end{itemize}
\end{definition}

Jusqu'à présent nous utilisions des labels sur la totalité des n\oe uds, par la suite nous en metteront uniquement sur ceux importants dont les irréductibles. Dans cet objectif, nous utiliserons des lettres (resp. chiffres) pour les $\vee$-irréductibles (resp. $\wedge$-irréductible), de même que les labels $\top$ et $\bot$. Sur l'exemple de la figure \ref{hasse_irr} les irréductibles sont représentés en noir, d'abord les $\vee$-irréductibles puis les $\wedge$-irréductibles.

\begin{figure}[H]
	\begin{minipage}{0.5\textwidth}
	\begin{center}
		\begin{tikzpicture}
			\node [bnode, label=below:{$\bot$}] (bot) at (0,0) {};
			\node [bnode, label=left:{1}] (1) at (-1, 1) {};
			\node [bnode, label=right:{2}] (2) at (1, 1) {};
			\node [bnode, label=left:{3}] (3) at (-1, 2) {};
			\node [wnode, label=right:{}] (4) at (1, 2) {};
			\node [wnode, label={$\top$}] (top) at (0, 3) {};
			
			\path [line] (bot) -- (1);
			\path [line] (bot) -- (2);
			\path [line] (1) -- (3);
			\path [line] (1) -- (4);
			\path [line] (2) -- (4);
			\path [line] (3) -- (top);
			\path [line] (4) -- (top);
		\end{tikzpicture}
	\end{center}
	\end{minipage}
	\begin{minipage}{0.5\textwidth}
	\begin{center}
		\begin{tikzpicture}
			\node [wnode, label=below:{$\bot$}] (bot) at (0,0) {};
			\node [wnode, label=left:{}] (1) at (-1, 1) {};
			\node [bnode, label=right:{$B$}] (2) at (1, 1) {};
			\node [bnode, label=left:{$A$}] (3) at (-1, 2) {};
			\node [bnode, label=right:{$C$}] (4) at (1, 2) {};
			\node [bnode, label={$\top$}] (top) at (0, 3) {};
			
			\path [line] (bot) -- (1);
			\path [line] (bot) -- (2);
			\path [line] (1) -- (3);
			\path [line] (1) -- (4);
			\path [line] (2) -- (4);
			\path [line] (3) -- (top);
			\path [line] (4) -- (top);
		\end{tikzpicture}
	\end{center}
	\end{minipage}
	\caption{Diagrammes de Hasse avec irréductibles}
	\label{hasse_irr}
\end{figure}

Il existe une catégorie de treillis appelée treillis distributif dont la particularité d'avoir la distributivité des opérations $\vee$ et $\wedge$. Pour tout $x, y, z \in T$ on a $x \vee (y \wedge z) = (x \wedge y) \vee (x \wedge z)$ et $x \wedge (y \vee z) = (x\vee y) \wedge (x \vee z)$. Cette propriété transparait à travers trois conditions équivalentes qui permettent d'attester ou non de la distributivité du treillis permettant ainsi d'utiliser la plus adaptée dans une situation donnée.

\begin{definition}[Treillis distributif]
Un treillis $(T, \leq, \vee, \wedge)$ est distributif si et seulement si au moins l'une des trois conditions équivalentes suivantes est vérifiée :
\begin{itemize}
	\item $(x \wedge y) \vee (x \wedge z) \vee (y \wedge z) = (x \vee y) \wedge (x \vee z) \wedge (y \vee z)$
	\item Le treillis ne contient ni $N_5$ ni $M_3$
	\item Le contexte réduit contient une seule relation flèche double par ligne et par colonne
\end{itemize}
\end{definition}

\begin{figure}[H]
	\begin{minipage}[t]{0.5\textwidth}
		\begin{center}
			\begin{tikzpicture}
				\node [wnode, label=below:{$\bot$}] (bot) at (0,0) {};
				\node [bnode, label=left:{1, $A$}] (1) at (-1, 1) {};
				\node [bnode, label=right:{2, $B$}] (2) at (1, 1) {};
				\node [bnode, label=left:{3, $C$}] (3) at (-1, 2) {};
				\node [wnode, label={$\top$}] (top) at (0, 3) {};

				\path [line] (bot) -- (1);
				\path [line] (bot) -- (2);
				\path [line] (1) -- (3);
				\path [line] (2) -- (top);
				\path [line] (3) -- (top);
			\end{tikzpicture}
		\end{center}
		\caption{$N_5$ (à déplacer)}
	\end{minipage}
	\begin{minipage}[t]{0.5\textwidth}
		\begin{center}
			\begin{tikzpicture}
				\node [wnode, label=below:{$\bot$}] (bot) at (0,0) {};
				\node [bnode, label=right:{1, $A$}] (1) at (-1.5, 1.5) {};
				\node [bnode, label=right:{2, $B$}] (2) at (0, 1.5) {};
				\node [bnode, label=right:{3, $C$}] (3) at (1.5, 1.5) {};
				\node [wnode, label={$\top$}] (top) at (0, 3) {};

				\path [line] (bot) -- (1);
				\path [line] (bot) -- (2);
				\path [line] (bot) -- (3);
				\path [line] (1) -- (top);
				\path [line] (2) -- (top);
				\path [line] (3) -- (top);
			\end{tikzpicture}
		\end{center}
		\caption{$M_3$ (à déplacer)}
	\end{minipage}
\end{figure}

\section{Graphe médian}

Un graphe est un diagramme composé de n\oe uds et d'arcs. Pour chaque triplets de n\oe uds, on définie l'ensemble des n\oe uds formant sur les plus courts chemins entre les éléments du triplet. Si pour tous les triplets du graphe cet ensemble ne contient qu'un unique n\oe ud, alors on dit que ce graphe est médian.