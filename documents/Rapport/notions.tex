\chapter{Notions nécessaires}

\section{Analyse de Concepts Formels}

L'Analyse Concepts Formels utilise des matrices binaires mettant en relation des objets avec des attributs, une matrice porte le nom de contexte et est défini par $C(J, M, I)$ où $J$ est l'ensemble des objets, $M$ l'ensemble des attributs et $I$ l'ensemble des relations entre les objets et les attributs.

\begin{definition}[Contexte]
Soit un contexte $C(J, M, I)$ dans lequel $J$ est l'ensemble des objets, $M$ l'ensemble des attributs et $I$ l'ensemble des relations entre les objets et les attributs.
\end{definition}

Nous définitions une opération, \guillemotleft{} prime \guillemotright{} pour les objets et attributs, notée \guillemotleft{} ' \guillemotright{}. Elle permet de passer des objets aux attributs et inversement. On lui donne un objet (resp. attribut) ou un ensemble d'objets (resp. attributs) et elle nous renvois l'ensemble des attributs (resp. objets) qui sont en correspondance.

\begin{definition}[Connexion de Gallois]
Pour $X \subseteq J$ et $Y \subseteq M$, on définie :
\begin{itemize}
	\item $X' = \{y \in M : (x, y) \in I$, $\forall x \in X\}$
	\item $Y' = \{x \in J : (x, y) \in I$, $\forall y \in Y\}$
\end{itemize}
\end{definition}

Dans la figure \ref{def_contexte}, nous avons $J = \{chat, chien, aigle, mouette, poule\}$, $M = \{griffes, ailes, nyclatope, vol\}$ et $I = \{(chat, griffes), (chat, nyctalope), (chien, griffes), (aigle, griffes), ...\}$. Nous avons entre autres pour les objets $chat' = \{griffes, nyctalope\}$ et $\{chat, aigle\}' = \{griffes\}$ et inversement pour les attributs $griffes' = \{chat, chien, aigle, poule\}$ et $\{griffes, ailes\}' = \{aigle, poule\}$.

\begin{figure}[H]
	\begin{center}
		\begin{tabular}{ l | c c c c }
			 & $griffes$ & $ailes$ & $nyctalope$ & $vol$ \\
			\hline
			$chat$ & x & & x & \\
			$chien$ & x & & & \\
			$aigle$ & x & x & & x \\
			$mouette$ & & x & & x \\
			$poule$ & x & x & & \\
		\end{tabular}
	\end{center}
	\caption{Contexte}
	\label{def_contexte}
\end{figure} 

Nous pouvont effectuer plusieurs opérations sur les contextes sans réelles pertes d'informations. La première est la clarification, cela consiste à ne garder que les lignes et les colonnes qui ne sont pas en doublon.

\begin{definition}[Contexte clarifié]
Soit un contexte clarifié $C(J, M, I)$ :
\begin{itemize}
	\item $\forall x1, x2 \in J$ si ${x1}' = {x2}'$ alors $x1 = x2$
	\item $\forall y1, y2 \in M$ si ${y1}' = {y2}'$ alors $y1 = y2$
\end{itemize}
\end{definition}

À partir du contexte non clarifié de la figure \ref{def_contexte_non_clarifie} nous obtenu le contexte clarifié de la figure \ref{def_contexte_clarifie} avec la suppression de la ligne $5$ qui est le doublon de la ligne $2$ et la colonne $e$ qui est le doublon de la $b$. Nous pouvons le voir à travers les opérations de prime, deux objets (resp. attributs) sont en doublon losqu'ils obtiennent le même résultat. Nous obtenons $2' = \{a\}$ et $5' = \{a\}$. Il est important de bien comprendre que malgré la suppression de lignes ou de colonnes dans cette opération, nous ne perdons aucune données.

\begin{figure}[H]
	\begin{minipage}[c]{0.5\textwidth}
	\begin{center}
		\begin{tabular}{ l | c c c c c }
			 & $a$ & $b$ & $c$ & $d$ & $e$ \\
			\hline
			$1$ & x & & & x & \\
			$2$ & x & & & & \\
			$3$ & x & x & x & & x \\
			$4$ & & x & x & & x \\
			$5$ & x & & & & \\
			$6$ & x & & x & & \\
		\end{tabular}
	\end{center}
	\caption{Contexte non clarifié}
	\label{def_contexte_non_clarifie}
	\end{minipage}
	\begin{minipage}[c]{0.5\textwidth}
	\begin{center}
		\begin{tabular}{ l | c c c c }
			 & $a$ & $b$ & $c$ & $d$ \\
			\hline
			$1$ & x & & & x \\
			$2$ & x & & & \\
			$3$ & x & x & x & \\
			$4$ & & x & x & \\
			$6$ & x & & x & \\
		\end{tabular}
	\end{center}
	\caption{Contexte clarifié}
	\label{def_contexte_clarifie}
	\end{minipage}
\end{figure}

La seconde opération est la plus utilisée et va plus loin dans la réduction de la taille du contexte sans perte de données, les contextes ainsi obtenus sont dit \guillemotleft{} contexte réduit \guillemotright \footnote{\guillemotleft{} standard context \guillemotright pour la documentation anglaise}. Le principe est toujours de supprimer les lignes et les colonnes dont on peut se passer, celles qui sont recalculables à partir de celles qu'on garde. On ne garde que les lignes et les colonnes qui ne sont pas l'intersection d'une ou plusieurs autres.

\begin{definition}[Contexte réduit]
Soit un contexte réduit $C(J, M, I)$ :
\begin{itemize}
	\item $\forall x \in C, \forall X \subseteq C$, si $x' = X'$ alors $x \in X$
	\item $\forall x \in M, \forall X \subseteq M$, si $x' = X'$ alors $x \in X$
\end{itemize}
\end{definition}

Dans la figure \ref{def_contexte_non_reduit} la ligne $2' = \{a\}$ est l'intersection des lignes $1' = \{a, d\}$ et $3' = \{a, b, c\}$ ou $1' = \{a, d\}$ et $6' = \{a, c\}$, on peut donc la supprimer. En revanche, nous n'avons aucune colonne dans ce cas ici. Lorsqu'on effectue la réduction d'un contexte, il est systématiquement sous entendu que l'opération de clarification est également effectuée.

\begin{figure}[H]
	\begin{minipage}[c]{0.5\textwidth}
	\begin{center}
		\begin{tabular}{ l | c c c c }
			 & $a$ & $b$ & $c$ & $d$ \\
			\hline
			$1$ & x & & & x \\
			$2$ & x & & & \\
			$3$ & x & x & x & \\
			$4$ & & x & x & \\
			$6$ & x & & x & \\
		\end{tabular}
	\end{center}
	\caption{Contexte non réduit}
	\label{def_contexte_non_reduit}
	\end{minipage}
	\begin{minipage}[c]{0.5\textwidth}
	\begin{center}
		\begin{tabular}{ l | c c c c }
			 & $a$ & $b$ & $c$ & $d$ \\
			\hline
			$1$ & x & & & x \\
			$3$ & x & x & x & \\
			$4$ & & x & x & \\
			$6$ & x & & x & \\
		\end{tabular}
	\end{center}
	\caption{Contexte réduit}
	\label{def_contexte_reduit}
	\end{minipage}
\end{figure}

Nous pouvons maintenant définir la notion de concept. Cela correspond aux rectangles maximum qui se trouve dans le contexte, regroupant tous les objets qui font parts du concept et tous les attributs qu'ils ont en commun. Un concept est l'ensemble des objets et attributs tel que tous les objets du concepts partagent les attributs du concepts et qu'il n'existe pas d'autre objet partageant les attributs du concept et pas d'autre attributs qui sont partagés par tous les objets du concept.

\begin{definition}[Concept]
Soit un contexte $C(J, M, I)$, un concept $a$, $X \subseteq a$, $X \subseteq J$, $Y \subseteq a$ et $Y \subseteq M$ :
\begin{itemize}
	\item $\forall x \in J$, $x \in X \Leftrightarrow x' \subseteq Y$
	\item $\forall y \in M$, $y \in Y \Leftrightarrow y' \subseteq X$
\end{itemize}
\end{definition}

Sur le contexte de la figure \ref{def_contexte_reduit} nous avons entre autre le concept contenant les objets 3 et 4 et les attributs $b$ et $c$ dont les correspondances forment un rectangle plein.

\bigbreak

\subsection{Treillis de concept}

\subsection{Relations flèches}

En plus des propriétés précédentes, un contexte peut contenir ce qu'on appelle des relations fléches qui ne viennent pas ajouter de l'information mais permettre une extraction plus rapide à porté humaine. Il en existe de trois types $\uparrow$, $\downarrow$, et $\updownarrow$.
\begin{description}
	\item[$1 \uparrow A$] : ssi A ne possède pas de sup une fois qu'on retire tous les éléments du filtre de 1.
	\item[$1 \downarrow A$] : ssi 1 ne possède pas d'inf une fois qu'on retire tous les éléments de l'idéal de A.
	\item[$1 \updownarrow A$] : ssi $1 \uparrow A$ et $1 \downarrow A$.
\end{description}

\begin{figure}[H]
	\begin{minipage}{0.4\textwidth}
	\begin{center}
		\begin{tabular}{ l | c c c c c }
			 & A & B & C & D & E \\
			\hline
			1 & x & x & x & x & $\updownarrow$ \\
			2 & x & $\updownarrow$ & x & x & x \\
			3 & x & x & x & $\updownarrow$ & \\
			4 & x & $\uparrow$ & $\updownarrow$ & x & x \\
			5 & x & x & $\updownarrow$ & $\uparrow$ & \\
			6 & $\updownarrow$ & x & & & \\
		\end{tabular}
		\caption{Contexte avec relations fléches}
	\end{center}
	\end{minipage}
	\begin{minipage}{0.8\textwidth}
		\begin{center}
			\begin{tikzpicture}
				\node [bnode, label=below:{bot}] (bot) at (0,0) {};
				\node [bnode, label=left:{1}] (1) at (-1,1) {};
				\node [bnode, label=right:{2}] (2) at (1,1) {};
				\node [bnode] (1_2) at (0,2) {};
				\node [bnode, label=left:{3}] (3) at (-2,2) {};
				\node [bnode, label=right:{4, E}] (4) at (2,2) {};
				\node [bnode, label=left:{5}] (5) at (-2,3) {};
				\node [bnode, label=right:{C}] (C) at (0,3) {};
				\node [bnode, label=right:{D}] (D) at (2,3) {};
				\node [bnode, label=left:{6, B}] (6) at (-2,4) {};
				\node [bnode, label=right:{A}] (A) at (0,4) {};
				\node [bnode, label={top}] (top) at (0,5) {};

				\path [line] (bot) -- (1);
				\path [line] (bot) -- (2);
				\path [line] (1) -- (3);
				\path [line] (1) -- (1_2);
				\path [line] (2) -- (4);
				\path [line] (2) -- (1_2);
				\path [line] (3) -- (5);
				\path [line] (3) -- (C);
				\path [line] (1_2) -- (C);
				\path [line] (1_2) -- (D);
				\path [line] (4) -- (D);
				\path [line] (5) -- (6);
				\path [line] (5) -- (A);
				\path [line] (C) -- (A);
				\path [line] (D) -- (A);
				\path [line] (6) -- (top);
				\path [line] (A) -- (top);
			\end{tikzpicture}
		\end{center}
		\caption{Treillis associé}
	\end{minipage}
\end{figure}

\section{Ensemble ordonné}

Un ensemble ordonné est un ensemble $P$ d'éléments avec une relation d'ordre $\leq$, qu'on note $(P, \leq)$. Pour le représenter, on utilise des diagrammes orientés du bas vers le  haut où chaque élément de $P$ est un point et chaque relation d'ordre est un arc dans le sens de lecture. Ce sont des diagrammes de Hasse.  Pour un ensemble $X \subseteq P$ on note $\uparrow \! X$ le filtre (resp. $\downarrow \! X$ l'idéal) de $X$. Pour un élément $x \in P$, on note $\uparrow \! x$ le filtre principal (resp. $\downarrow \! x$ l'idéal principal) de $x$ tel que pour $\uparrow \! x = {y}$ (resp. $\downarrow \! x = {y}$) on a $x \leq y$ (resp. $y \leq x$). Dans la figure \ref{hasse_filtres_ideaux} nous avons $\uparrow \! a = {a, b, c}$ et $\downarrow \! b = {a, b}$.

\begin{figure}[H]
	\begin{minipage}{0.3\textwidth}
	\begin{center}
		\begin{tikzpicture}
			\node [wnode, label=below:{$a$}] (a) at (-1, 0) {};
			\node [wnode, label=left:{$b$}] (b) at (-2, 1) {};
			\node [wnode, label=right:{$c$}] (c) at (0, 1) {};
			
			\path [line] (a) -- (b);
			\path [line] (a) -- (c);
		\end{tikzpicture}
	\end{center}
	\end{minipage}
	\begin{minipage}{0.3\textwidth}
	\begin{center}
		\begin{tikzpicture}
			\node [bnode, label=below:{$\uparrow \! a = \{a, b, c\}$}] (a) at (-1, 0) {};
			\node [bnode, label=left:{$b$}] (b) at (-2, 1) {};
			\node [bnode, label=right:{$c$}] (c) at (0, 1) {};
			
			\path [line] (a) -- (b);
			\path [line] (a) -- (c);
			\draw [very thick, dotted] plot [smooth, tension=0.5] coordinates {(-2.2, 1) (-1, -0.2) (0.2, 1)};
		\end{tikzpicture}
	\end{center}
	\end{minipage}
	\begin{minipage}{0.3\textwidth}
	\begin{center}
		\begin{tikzpicture}
			\node [bnode, label=below:{$a$}] (a) at (-1, 0) {};
			\node [bnode, label={$\downarrow \! b = \{a, b\}$}] (b) at (-2, 1) {};
			\node [wnode, label=right:{$c$}] (c) at (0, 1) {};
			
			\path [line] (a) -- (b);
			\path [line] (a) -- (c);
			\draw [very thick, dotted] plot [smooth, tension=0.5] coordinates {(-2.2, 0) (-2, 1.2) (-0.8, 0)};
		\end{tikzpicture}
	\end{center}
	\end{minipage}
	\caption{Diagrammes de Hasse avec filtres et idéaux}
	\label{hasse_filtres_ideaux}
\end{figure}

\section{Treillis}

Un treillis  est un ensemble ordonné $(T, \leq)$ à lequel on ajoute deux opérateurs, on le note $(T, \leq, \vee, \wedge)$. Le premier est celui de supremum ou borne supérieure noté $\vee$.  Pour trois éléments $x, y, z \in P$ tel que $z \leq x$ et $z \leq y$ on peut dire que $x \vee y = z$. Le second est celui d'infimum ou borne inférieure noté $\wedge$. Et de la même façon pour trois éléments $x, y, z \in P$ tel que $x \leq z$ et $y \leq z$ on peut dire que $x \wedge y = z$. En reprenant l'exemple de la figure \ref{hasse_sup_inf}, nous obtenons $b \vee c = a$ et $b \wedge c = d$. Par abus de langage on utilise souvent les termes de sup et d'inf. Lorsqu'un élément est la borne supérieure (resp. inférieure) de tous les autres éléments on dit qu'il ferme le treillis par les supremums (resp. infimums), on note cet élément $\top$ (resp.$\bot$). Et lorsqu'un treillis est fermé à la fois par les supremums et par les infimums on dit que c'est un treillis fermé défini comme le cas fini. Sur notre exemple en figure \ref{hasse_sup_inf} nous avons $\top = d$ et $\bot = a$.

\begin{figure}[H]
	\begin{minipage}{0.5\textwidth}
	\begin{center}
		\begin{tikzpicture}
			\node [bnode, label=below:{$a = b \vee c$}] (a) at (0, 0) {};
			\node [bnode, label=left:{$b$}] (b) at (-1, 1) {};
			\node [bnode, label=right:{$c$}] (c) at (1, 1) {};
			\node [wnode, label={$d$}] (d) at (0, 2) {};
			
			\path [line] (a) -- (b);
			\path [line] (a) -- (c);
			\path [line] (b) -- (d);
			\path [line] (c) -- (d);
		\end{tikzpicture}
	\end{center}
	\end{minipage}
	\begin{minipage}{0.5\textwidth}
	\begin{center}
		\begin{tikzpicture}
			\node [wnode, label=below:{$a$}] (a) at (0, 0) {};
			\node [bnode, label=left:{$b$}] (b) at (-1, 1) {};
			\node [bnode, label=right:{$c$}] (c) at (1, 1) {};
			\node [bnode, label={$d = b \wedge c$}] (d) at (0, 2) {};
			
			\path [line] (a) -- (b);
			\path [line] (a) -- (c);
			\path [line] (b) -- (d);
			\path [line] (c) -- (d);
		\end{tikzpicture}
	\end{center}
	\end{minipage}
	\caption{Diagrammes de Hasse avec supremum et infimum}
	\label{hasse_sup_inf}
\end{figure}

Un élément qui est pas le supremum (resp. infimum) d'autres éléments est un $\vee$-irréductible (resp. $\wedge$-irréductible) et on note l'ensemble de ces élément $J(T)$ (resp. $M(T)$). Bien que le $\top$ et le $\bot$ soient des irréductibles, il est commun de ne pas toujours les prendre en considération, nous les prendront dans ce rapport, il sera précisé lorsqu'ils ne le seront pas\todo{ou inversement}. Jusqu'à présent nous utilisions des labels sur la totalité des points, par la suite nous en metteront uniquement sur ceux importants dont les irréductibles. Dans cet objectif, nous utiliserons des lettres (resp. chiffres) pour les $\vee$-irréductibles (resp. $\wedge$-irréductible) et le label \guillemotleft{} top \guillemotright{} (resp. \guillemotleft{} bot \guillemotright{}) pour $\top$ (resp. $\bot$). Sur l'exemple de la figure \ref{hasse_irr} les irréductibles sont représentés en noir, d'abord les $\vee$-irréductibles puis les $\wedge$-irréductibles.

\begin{figure}[H]
	\begin{minipage}{0.5\textwidth}
	\begin{center}
		\begin{tikzpicture}
			\node [bnode, label=below:{$\bot$}] (bot) at (0,0) {};
			\node [bnode, label=left:{1}] (1) at (-1, 1) {};
			\node [bnode, label=right:{2}] (2) at (1, 1) {};
			\node [bnode, label=left:{3}] (3) at (-1, 2) {};
			\node [wnode, label=right:{}] (4) at (1, 2) {};
			\node [wnode, label={$\top$}] (top) at (0, 3) {};
			
			\path [line] (bot) -- (1);
			\path [line] (bot) -- (2);
			\path [line] (1) -- (3);
			\path [line] (1) -- (4);
			\path [line] (2) -- (4);
			\path [line] (3) -- (top);
			\path [line] (4) -- (top);
		\end{tikzpicture}
	\end{center}
	\end{minipage}
	\begin{minipage}{0.5\textwidth}
	\begin{center}
		\begin{tikzpicture}
			\node [wnode, label=below:{$\bot$}] (bot) at (0,0) {};
			\node [wnode, label=left:{}] (1) at (-1, 1) {};
			\node [bnode, label=right:{$B$}] (2) at (1, 1) {};
			\node [bnode, label=left:{$A$}] (3) at (-1, 2) {};
			\node [bnode, label=right:{$C$}] (4) at (1, 2) {};
			\node [bnode, label={$\top$}] (top) at (0, 3) {};
			
			\path [line] (bot) -- (1);
			\path [line] (bot) -- (2);
			\path [line] (1) -- (3);
			\path [line] (1) -- (4);
			\path [line] (2) -- (4);
			\path [line] (3) -- (top);
			\path [line] (4) -- (top);
		\end{tikzpicture}
	\end{center}
	\end{minipage}
	\caption{Diagrammes de Hasse avec irréductibles}
	\label{hasse_irr}
\end{figure}

\subsection{Treillis distributif}

Un treillis est dis distributif s'il vérifie l'une des conditions suivantes :
\begin{itemize}
	\item $(x \wedge y) \vee (x \wedge z) \vee (y \wedge z) = (x \vee y) \wedge (x \vee z) \wedge (y \vee z)$
	\item Le treillis ne contient ni $N_5$ ni $M_3$
	\item Le contexte standard contient une seule relation flèche double par ligne et par colonne
\end{itemize}

\begin{figure}[H]
	\begin{minipage}[t]{0.5\textwidth}
		\begin{center}
			\begin{tikzpicture}
				\node [wnode, label=below:{bot}] (bot) at (0,0) {};
				\node [bnode, label=left:{1, A}] (1) at (-1, 1) {};
				\node [bnode, label=right:{2, B}] (2) at (1, 1) {};
				\node [bnode, label=left:{3, C}] (3) at (-1, 2) {};
				\node [wnode, label={top}] (top) at (0, 3) {};

				\path [line] (bot) -- (1);
				\path [line] (bot) -- (2);
				\path [line] (1) -- (3);
				\path [line] (2) -- (top);
				\path [line] (3) -- (top);
			\end{tikzpicture}
		\end{center}
		\caption{$N_5$}
	\end{minipage}
	\begin{minipage}[t]{0.5\textwidth}
		\begin{center}
			\begin{tikzpicture}
				\node [wnode, label=below:{bot}] (bot) at (0,0) {};
				\node [bnode, label=right:{1, A}] (1) at (-1.5, 1.5) {};
				\node [bnode, label=right:{2, B}] (2) at (0, 1.5) {};
				\node [bnode, label=right:{3, C}] (3) at (1.5, 1.5) {};
				\node [wnode, label={top}] (top) at (0, 3) {};

				\path [line] (bot) -- (1);
				\path [line] (bot) -- (2);
				\path [line] (bot) -- (3);
				\path [line] (1) -- (top);
				\path [line] (2) -- (top);
				\path [line] (3) -- (top);
			\end{tikzpicture}
		\end{center}
		\caption{$M_3$}
	\end{minipage}
\end{figure}

De plus, un treillis distributif dispose de quelques propriétés supplémentaires. Tous ses sous-treillis sont également distributifs. Il peut être également être considéré comme un graphe médian : pour tout ensemble de trois n\oe uds, il existe un unique n\oe ud d'intersection entre les chemins les plus courts.

\section{Graphe médian}