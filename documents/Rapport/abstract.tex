\renewcommand{\abstractnamefont}{\normalfont\Large\bfseries}
%\renewcommand{\abstracttextfont}{\normalfont\Huge}

\begin{abstract}
\hskip7mm

\begin{spacing}{1.3}

L'étude de filiation des espèces, phylogénie est un domaine de recherche utilisant des arbres parcimonieux. Par le passé, H.J. Bandelt \cite{cla2018} a mis en avant l'avantage qu'il y a à gagner à encoder l'ensemble des informations dans un unique graphe médian regroupant l'ensemble de ces arbres en même temps. Puis cela à été au tour de U. Priss \cite{MedianConceptLattices} de présenter l'avantage de cette vision qui permettrait de donner à la phylogénie accès aux méthodes de l'analyse de concepts formels, tout en présentant un point de départ de méthode permettant de passer d'un arbre quelconque sous forme de treillis qui a été poursuivis en un algorithme pour les cas simple par l'équipe ORPAILLEUR\cite{cla2018}. Et mon travail présenté dans ce dossier va porter sur le prolongement de cet algorithme pour sa mise en place dans un programme et son amélioration pour la prise en compte de cas plus complexes.

\end{spacing}
\end{abstract}
