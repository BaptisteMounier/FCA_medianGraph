	\documentclass{beamer}

\usepackage[french]{babel}
\usepackage[T1]{fontenc}
\usepackage[utf8x]{inputenc}

\usetheme{Warsaw}
%\usetheme{Berkeley}
%\usetheme{Antibes}

\usepackage{subfig}
\usepackage{tikz}
\usetikzlibrary{shapes,arrows, positioning}
\usetikzlibrary{shadows}

% tikz rules
\tikzstyle{bnode} = [circle, fill, draw, inner sep = 2pt]
\tikzstyle{wnode} = [circle, draw, inner sep = 2pt]
\tikzstyle{line} = [draw]
\tikzstyle{block2} = [rectangle, draw, text width = 5em, text centered, minimum height = 3em]
\tikzstyle{decision} = [diamond, draw, fill = green!20, text width = 6em, text badly centered, inner sep = 0pt]
\tikzstyle{block} = [rectangle, draw, fill = blue!20, text width = 6em, text centered, rounded corners, minimum height = 3em]
\tikzstyle{algorithm} = [rectangle, draw, fill = green!20, text width = 6em, text centered, rounded corners, minimum height = 3em]
\tikzstyle{inout} = [rectangle, draw, fill = red!20, text width = 6em, text centered, minimum height = 3em]
\tikzstyle{cloud} = [ellipse, draw, fill = red!20, minimum height=2em]
\tikzstyle{line2} = [draw, -latex']
\tikzstyle{line3} = [draw, dashed, -latex']
\tikzstyle{line4} = [draw, dotted, -latex']
\tikzstyle{my below of} = [below=of #1.south]
\tikzstyle{my right of} = [right=of #1.east]
\tikzstyle{my left of} = [left=of #1.west]
\tikzstyle{my above of} = [above=of #1.north]

\defbeamertemplate*{footline}{shadow theme}
{%
  \leavevmode%
  \hbox{\begin{beamercolorbox}[wd=.5\paperwidth,ht=2.5ex,dp=1.125ex,leftskip=.3cm plus1fil,rightskip=.3cm]{author in head/foot}%
    \usebeamerfont{author in head/foot}\insertframenumber\,/\,\inserttotalframenumber\hfill\insertshortauthor
  \end{beamercolorbox}%
  \begin{beamercolorbox}[wd=.5\paperwidth,ht=2.5ex,dp=1.125ex,leftskip=.3cm,rightskip=.3cm plus1fil]{title in head/foot}%
    \usebeamerfont{title in head/foot}\insertshorttitle%
  \end{beamercolorbox}}%
  \vskip0pt%
}

%\title[Graphes médians]{Production de graphes médians pour la phylogénie}
\title{Graphes médians, FCA et phylogénie}
\subtitle{Master 2 Sciences Cognitives et Médias Numériqes}
\author{Baptiste Mounier}
\institute{UFR Mathématiques et Informatique \and LORIA}
\date{09 avril 2018 - 08 août 2018}

%\usecolortheme{Magpie}

\AtBeginSection[]{
	\begin{frame}
		\frametitle{Sommaire}
%		\tableofcontents[currentsection, hideothersubsections]
		\tableofcontents[sectionstyle=show/hide,subsectionstyle=show/show/hide ]
	\end{frame} 
}

\begin{document}

\begin{frame}[plain]
	\titlepage
\end{frame}

\begin{frame}{Sommaire}
	\tableofcontents[hideallsubsections]
\end{frame}


\section{Contexte}

\subsection{LORIA}

\begin{frame}
	\frametitle{LORIA}
	\begin{itemize}
		\item {\bf L}aboratoire L{\bf o}rrain de {\bf R}echerche en {\bf I}nforatique et ses {\bf A}pplications
		\item UMR 7503 : CNRS, Université de Lorraine et Inria
		\item Création en 1997
		\item Fédération Charles Hermite
		\item 5 départements et 28 équipes - 400 personnes
	\end{itemize}
\end{frame}

\subsection{ORPAILLEUR}
\begin{frame}
	\frametitle{ORPAILLEUR}
	\begin{itemize}
		\item Équipe ORPAILLEUR
		\item 4ème département : Traitement automatique des langues et des connaissances
		\item Miguel Couceiro, Alian Gély, Amedeo Napoli
		\item Exploration de connaissances dans des bases de données
	\end{itemize}
\end{frame}

\section{Motivations}

\subsection{Phylogénie}

\begin{frame}
	\frametitle{Données : matrice binaires espèce/caractéristique}
	\begin{center}
		\begin{tabular}{ l | c c c c }
			 & $griffes$ & $ailes$ & $nyctalope$ & $vol$ \\
			\hline
			$chat$ & x & & x & \\
			$chien$ & x & & & \\
			$aigle$ & x & x & & x \\
			$mouette$ & & x & & x \\
			$poule$ & x & x & & \\
		\end{tabular}
	\end{center}
\end{frame}

\begin{frame}
	\frametitle{Représentation : arbres}
	\begin{center}
	\only<1| handout:1> {
		\begin{tikzpicture}
			\node [bnode, label=below:{chat}] (chat) at (0, 0) {};
			\node [bnode, label=below:{chien}] (chien) at (2, 0) {};
			\node [bnode, label=below:{poule}] (poule) at (4, 0) {};
			\node [bnode, label=below:{mouette}] (mouette) at (6, 0) {};
			\node [bnode, label=below:{aigle}] (aigle) at (8, 0) {};
			\node [wnode, label=left:{nyctalope}] (nyctalope) at (0, 2) {};
			\node [wnode, label=left:{griffes}] (griffes) at (2, 3) {};
			\node [wnode, label=right:{ailes}] (ailes) at (4, 3) {};
			\node [wnode, label=right:{griffes}] (griffes3) at (4, 2) {};
			\node [wnode, label={vol}] (vol) at (6, 2) {};
			\node [wnode, label=right:{griffes}] (griffes2) at (8, 1) {};
			\node [wnode, label={}] (top) at (3, 4) {};
			
			\path [line] (chat) -- (nyctalope);
			\path [line] (nyctalope) -- (griffes);
			\path [line] (chien) -- (griffes);
			\path [line] (poule) -- (griffes3);
			\path [line] (griffes3) -- (ailes);
			\path [line] (mouette) -- (vol);
			\path [line] (vol) -- (ailes);
			\path [line] (aigle) -- (griffes2);
			\path [line] (griffes2) -- (vol);
			\path [line] (griffes) -- (top);
			\path [line] (ailes) -- (top);
		\end{tikzpicture}
	}\only<2| handout:2> {
		\begin{tikzpicture}
			\node [bnode, label=below:{chat}] (chat) at (0, 0) {};
			\node [bnode, label=below:{chien}] (chien) at (1, 0) {};
			\node [bnode, label=below:{poule}] (poule) at (2, 0) {};
			\node [bnode, label=below:{aigle}] (aigle) at (3, 0) {};
			\node [bnode, label=below:{mouette}] (mouette) at (4.2, 0) {};
			\node [wnode, label=left:{nyctalope}] (nyctalope) at (0, 2) {};
			\node [wnode, label=left:{griffes}] (griffes) at (1, 3) {};
			\node [wnode, label=left:{ailes}] (ailes) at (2, 2) {};
			\node [wnode, label=left:{vol}] (vol) at (3, 1) {};
			\node [wnode, label=right:{ailes}] (ailes2) at (4.2, 3) {};
			\node [wnode, label=right:{vol}] (vol2) at (4.2, 2) {};
			\node [wnode, label={}] (top) at (2, 4) {};
			
			\path [line] (chat) -- (nyctalope);
			\path [line] (nyctalope) -- (griffes);
			\path [line] (chien) -- (griffes);
			\path [line] (ailes) -- (griffes);
			\path [line] (poule) -- (ailes);
			\path [line] (vol) -- (ailes);
			\path [line] (aigle) -- (vol);
			\path [line] (griffes) -- (top);
			\path [line] (mouette) -- (vol2);
			\path [line] (vol2) -- (ailes2);
			\path [line] (ailes2) -- (top);
		\end{tikzpicture}
	}
	\end{center}
\end{frame}

\subsection{Graphes médians}

\begin{frame}
	\frametitle{Un outil : Graphes médians}
		\begin{columns} % align columns
			\begin{column}{.3\textwidth}
				\begin{itemize}
					\item Hans-Jürgen Bandelt
					\item Graphe médian
					\item Intégralité des arbres parcimonieux
				\end{itemize}
			\end{column}%
			\begin{column}{.6\textwidth}
		\begin{tikzpicture}
			\node [bnode, label=below:{chat, nyctalope}] (chat) at (0, 0) {};
			\node [wnode, label=left:{chien, griffes}] (chien) at (0, 2) {};
			\node [wnode, label=below:{poule}] (poule) at (1, 1) {};
			\node [bnode, label=below:{aigle}] (aigle) at (2, 0) {};
			\node [wnode, label=right:{mouette, vol}] (mouette) at (2, 1) {};
			\node [wnode, label=right:{ailes}] (ailes) at (2, 2) {};
			\node [wnode, label={}] (top) at (1, 3) {};
			
			\path [line] (chat) -- (chien);
			\path [line] (chien) -- (top);
			\path [line] (aigle) -- (poule);
			\path [line] (aigle) -- (mouette);
			\path [line] (mouette) -- (ailes);
			\path [line] (poule) -- (chien);
			\path [line] (poule) -- (ailes);
			\path [line] (ailes) -- (top);
			\path [line] (chien) -- (top);
		\end{tikzpicture}
			\end{column}
		\end{columns}
\end{frame}

\subsection{Analyse de Concepts Formels}

\begin{frame}
	\frametitle{Une approche : Analyse de Concepts Formels}
	\begin{itemize}
		\item Les matrices binaires sont centrales
		\item Catégorisation d'individus suivant leurs caractéristiques
	\end{itemize}
\end{frame}

\begin{frame}
	\frametitle{Une approche : Analyse de Concepts Formels}
	\begin{center}
		\begin{tikzpicture}
			\node [wnode, label=below:{$\bot$}] (bot) at (0, 0) {};
			\node [wnode, label=left:{chat, nyctalope}] (chat) at (-1, 1) {};
			\node [wnode, label=right:{aigle}] (aigle) at (1, 1) {};
			\node [wnode, label=right:{poule}] (poule) at (0, 2) {};
			\node [wnode, label=right:{mouette, vol}] (mouette) at (2, 2) {};
			\node [wnode, label=left:{chien, griffes}] (chien) at (-1, 3) {};
			\node [wnode, label=right:{ailes}] (ailes) at (1, 3) {};
			\node [wnode, label={$\top$}] (top) at (0, 4) {};
			
			\path [line] (bot) -- (chat);
			\path [line] (bot) -- (aigle);
			\path [line] (chat) -- (chien);
			\path [line] (chien) -- (top);
			\path [line] (aigle) -- (poule);
			\path [line] (aigle) -- (mouette);
			\path [line] (mouette) -- (ailes);
			\path [line] (poule) -- (chien);
			\path [line] (poule) -- (ailes);
			\path [line] (ailes) -- (top);
			\path [line] (chien) -- (top);
		\end{tikzpicture}
	\end{center}
\end{frame}

\begin{frame}
	\frametitle{Une approche : Analyse de Concepts Formels}
	\begin{itemize}
		\item Uta Priss
		\item Outils et la communauté de la FCA pour la phylogénie
		\item Liens entre treillis distributifs et graphes médians
		\item Proposition de méthode : transformer un treillis quelconque en un graphe médian
	\end{itemize}
\end{frame}

\begin{frame}
	\frametitle{Une approche : Analyse de Concepts Formels}
	\begin{itemize}
		\item Orpailleur (Yacine Namir)
		\item Implémentation d'une solution
		\item Problèmes sur certains types de cas
	\end{itemize}
\end{frame}

\section{Notions nécessaires}

\subsection{Analyse de Concepts Formels}

\begin{frame}
	\frametitle{Contexte $C(O, A, I)$}
	$O = \{chat, chien, aigle, mouette, poule\}$\\
	$A = \{griffes, ailes, nyctalope, vol\}$\\
	$I = \{(chat, griffes), (chat, nyctalope), (chien, griffes), ...\}$
	\begin{center}
		\begin{tabular}{ l | c c c c }
			 & $griffes$ & $ailes$ & $nyctalope$ & $vol$ \\
			\hline
			$chat$ & x & & x & \\
			$chien$ & x & & & \\
			$aigle$ & x & x & & x \\
			$mouette$ & & x & & x \\
			$poule$ & x & x & & \\
		\end{tabular}
	\end{center}
\end{frame}

\begin{frame}
	\frametitle{Connexion de Gallois}
		\only<1| handout:1> {
			\begin{definition}
				Pour $X \subseteq O$ et $Y \subseteq A$, on définie :
				\begin{itemize}
					\item $X' = \{y \in A : (x, y) \in I$, $\forall x \in X\}$
					\item $Y' = \{x \in O : (x, y) \in I$, $\forall y \in Y\}$
				\end{itemize}
			\end{definition}
		}\only<2| handout:2> {
			\begin{center}
				\begin{tabular}{ l | c c c c }
					 & $griffes$ & $ailes$ & $nyctalope$ & $vol$ \\
					\hline
					$chat$ & x & & x & \\
					$chien$ & x & & & \\
					$aigle$ & x & x & & x \\
					$mouette$ & & x & & x \\
					$poule$ & x & x & & \\
				\end{tabular}
			\end{center}
			$\{chat, aigle\}' = \{griffes\}$\\
			$\{griffes, ailes\}' = \{aigle, poule\}$
		}
\end{frame}

\begin{frame}
	\frametitle{Contexte clarifié}
	\only<1| handout:1> {
		\begin{itemize}
			\item À partir d'un contexte
			\item Aucune redondance
			\item Conservation de la structure
			\item Perte uniquement de labels
		\end{itemize}
		\begin{definition}
			Un contexte $C(O, A, I)$ est clarifié si et seulement si:
			\begin{itemize}
				\item $\forall x1, x2 \in O$ si ${x1}' = {x2}'$ alors $x1 = x2$
				\item $\forall y1, y2 \in A$ si ${y1}' = {y2}'$ alors $y1 = y2$
			\end{itemize}
		\end{definition}
	}\only<2| handout:2> {
		\begin{columns}[T] % align columns
			\begin{column}{.48\textwidth}
				\begin{center}
				\begin{tabular}{ l | c c c c c }
					 & $a$ & $b$ & $c$ & $d$ & $e$ \\
					\hline
					$1$ & x & & & x & \\
					$2$ & x & & & & \\
					$3$ & x & x & x & & x \\
					$4$ & & x & x & & x \\
					$5$ & x & & & & \\
					$6$ & x & & x & & \\
				\end{tabular}
				\end{center}
			\end{column}%
			\hfill%
			\begin{column}{.48\textwidth}
				\begin{center}
				\begin{tabular}{ l | c c c c }
					 & $a$ & $b$ & $c$ & $d$ \\
					\hline
					$1$ & x & & & x \\
					$2$ & x & & & \\
					$3$ & x & x & x & \\
					$4$ & & x & x & \\
					$6$ & x & & x & \\
				\end{tabular}
				\end{center}
			\end{column}%
		\end{columns}
	}
\end{frame}

\begin{frame}
	\frametitle{Contexte réduit}
	\only<1| handout:1> {
		\begin{itemize}
			\item À partir d'un contexte clarifié
			\item Aucune information recalculable
			\item Conservation de la structure
			\item Perte uniquement de labels
		\end{itemize}
		\begin{definition}
			Un contexte clarifié $C(O, A, I)$ est réduit si et seulement si :
			\begin{itemize}
				\item $\forall x \in O, \forall X \subseteq O$, si $x' = X'$ alors $x \in X$
				\item $\forall y \in A, \forall Y \subseteq A$, si $y' = Y'$ alors $y \in Y$
			\end{itemize}
		\end{definition}
	}\only<2| handout:2> {
		\begin{columns}[T] % align columns
			\begin{column}{.48\textwidth}
				\begin{center}
				\begin{tabular}{ l | c c c c }
					 & $a$ & $b$ & $c$ & $d$ \\
					\hline
					$1$ & x & & & x \\
					$2$ & x & & & \\
					$3$ & x & x & x & \\
					$4$ & & x & x & \\
					$6$ & x & & x & \\
				\end{tabular}
				\end{center}
			\end{column}%
			\hfill%
			\begin{column}{.48\textwidth}
				\begin{center}
				\begin{tabular}{ l | c c c c }
					 & $a$ & $b$ & $c$ & $d$ \\
					\hline
					$1$ & x & & & x \\
					$3$ & x & x & x & \\
					$4$ & & x & x & \\
					$6$ & x & & x & \\
				\end{tabular}
				\end{center}
			\end{column}%
		\end{columns}
	}
\end{frame}

\begin{frame}
	\frametitle{Concept}
	\begin{columns}[T] % align columns
		\begin{column}{.48\textwidth}
			\begin{definition}
				Soit un concept $c$, $X \subseteq O$ et $Y \subseteq A$ :
				\begin{itemize}
					\item $c = \{X, Y\}$
					\item $\forall x \in O$, $x \in X \Leftrightarrow x' \subseteq Y$
					\item $\forall y \in A$, $y \in Y \Leftrightarrow y' \subseteq X$
				\end{itemize}
			\end{definition}
		\end{column}%
		\hfill%
		\begin{column}{.48\textwidth}
			\begin{center}
			\begin{tabular}{ l | c c c c }
				 & $a$ & $b$ & $c$ & $d$ \\
				\hline
				$1$ & x & & & x \\
				$3$ & x & x & x & \\
				$4$ & & x & x & \\
				$6$ & x & & x & \\
			\end{tabular}
			\end{center}
		\end{column}%
	\end{columns}
\end{frame}

\begin{frame}
	\frametitle{Treillis de concept}
	\begin{columns}[T] % align columns
		\begin{column}{.4\textwidth}
			\begin{center}
			\begin{tabular}{ l | c c c c c }
				 & $a$ & $b$ & $c$ & $d$ & $e$ \\
				\hline
				1 & x & x & x & x &  \\
				2 & x & & x & x & x \\
				3 & x & x & x & & \\
				4 & x & & & x & x \\
				5 & x & x & & & \\
				6 & & x & & & \\
			\end{tabular}
			\end{center}
		\end{column}%
		\hfill%
		\begin{column}{.6\textwidth}
			\begin{tikzpicture}
				\node [wnode, label=below:{$\bot$}] (bot) at (0,0) {};
				\node [wnode, label=left:{1}] (1) at (-1,1) {};
				\node [wnode, label=right:{2}] (2) at (1,1) {};
				\node [wnode] (1_2) at (0,2) {};
				\node [wnode, label=left:{3}] (3) at (-2,2) {};
				\node [wnode, label=right:{4, $e$}] (4) at (2,2) {};
				\node [wnode, label=left:{5}] (5) at (-2,3) {};
				\node [wnode, label=right:{$c$}] (C) at (0,3) {};
				\node [wnode, label=right:{$d$}] (D) at (2,3) {};
				\node [wnode, label=left:{6, $b$}] (6) at (-2,4) {};
				\node [wnode, label=right:{$a$}] (A) at (0,4) {};
				\node [wnode, label={$\top$}] (top) at (0,5) {};
				
				\path [line] (bot) -- (1);
				\path [line] (bot) -- (2);
				\path [line] (1) -- (3);
				\path [line] (1) -- (1_2);
				\path [line] (2) -- (4);
				\path [line] (2) -- (1_2);
				\path [line] (3) -- (5);
				\path [line] (3) -- (C);
				\path [line] (1_2) -- (C);
				\path [line] (1_2) -- (D);
				\path [line] (4) -- (D);
				\path [line] (5) -- (6);
				\path [line] (5) -- (A);
				\path [line] (C) -- (A);
				\path [line] (D) -- (A);
				\path [line] (6) -- (top);
				\path [line] (A) -- (top);
			\end{tikzpicture}
		\end{column}%
	\end{columns}
\end{frame}

\begin{frame}
	\frametitle{Relations flèches}
	\begin{columns}[T] % align columns
		\begin{column}{.4\textwidth}
			\begin{center}
			\begin{tabular}{ l | c c c c c }
				 & $a$ & $b$ & $c$ & $d$ & $e$ \\
				\hline
				1 & x & x & x & x & $\updownarrow$ \\
				2 & x & $\updownarrow$ & x & x & x \\
				3 & x & x & x & $\updownarrow$ & \\
				4 & x & $\uparrow$ & $\updownarrow$ & x & x \\
				5 & x & x & $\updownarrow$ & $\uparrow$ & \\
				6 & $\updownarrow$ & x & & & \\
			\end{tabular}
			\end{center}
		\end{column}%
		\hfill%
		\begin{column}{.6\textwidth}
			\begin{tikzpicture}
				\node [wnode, label=below:{$\bot$}] (bot) at (0,0) {};
				\node [wnode, label=left:{1}] (1) at (-1,1) {};
				\node [wnode, label=right:{2}] (2) at (1,1) {};
				\node [wnode] (1_2) at (0,2) {};
				\node [wnode, label=left:{3}] (3) at (-2,2) {};
				\node [wnode, label=right:{4, $e$}] (4) at (2,2) {};
				\node [wnode, label=left:{5}] (5) at (-2,3) {};
				\node [wnode, label=right:{$c$}] (C) at (0,3) {};
				\node [wnode, label=right:{$d$}] (D) at (2,3) {};
				\node [wnode, label=left:{6, $b$}] (6) at (-2,4) {};
				\node [wnode, label=right:{$a$}] (A) at (0,4) {};
				\node [wnode, label={$\top$}] (top) at (0,5) {};
				
				\path [line] (bot) -- (1);
				\path [line] (bot) -- (2);
				\path [line] (1) -- (3);
				\path [line] (1) -- (1_2);
				\path [line] (2) -- (4);
				\path [line] (2) -- (1_2);
				\path [line] (3) -- (5);
				\path [line] (3) -- (C);
				\path [line] (1_2) -- (C);
				\path [line] (1_2) -- (D);
				\path [line] (4) -- (D);
				\path [line] (5) -- (6);
				\path [line] (5) -- (A);
				\path [line] (C) -- (A);
				\path [line] (D) -- (A);
				\path [line] (6) -- (top);
				\path [line] (A) -- (top);
			\end{tikzpicture}
		\end{column}%
	\end{columns}
\end{frame}

\subsection{Ensemble ordonné}

\begin{frame}
	\frametitle{Ensemble ordonné}
	\begin{definition}
		Soit un ensemble ordonné $(P, \leq)$, $X \subseteq P$ et $Y \subseteq P$ :
		\begin{itemize}
			\item $\forall x, y \in P$, on a $x \leq y$ ou $y \leq x$
			\item $\uparrow \! X = Y \Rightarrow x \leq y$ $\forall x \in X, \forall y \in Y$
			\item $\downarrow \! X = Y \Rightarrow y \leq x$ $\forall x \in X, \forall y \in Y$
		\end{itemize}
	\end{definition}
	
	\begin{columns}[T]
		\begin{column}{.2\textwidth}
			\begin{center}
			\begin{tikzpicture}
				\node [wnode, label=below:{$a$}] (a) at (-1, 0) {};
				\node [wnode, label=left:{$b$}] (b) at (-2, 1) {};
				\node [wnode, label=right:{$c$}] (c) at (0, 1) {};
				\path [line] (a) -- (b);
				\path [line] (a) -- (c);
			\end{tikzpicture}
			\end{center}
		\end{column}
		\begin{column}{.2\textwidth}
			\begin{center}
			\begin{tikzpicture}
				\node [bnode, label=below:{$\uparrow \! a = \{a, b, c\}$}] (a) at (-1, 0) {};
				\node [bnode, label=left:{$b$}] (b) at (-2, 1) {};
				\node [bnode, label=right:{$c$}] (c) at (0, 1) {};
				\path [line] (a) -- (b);
				\path [line] (a) -- (c);
				\draw [very thick, dotted] plot [smooth, tension=0.5] coordinates {(-2.2, 1) (-1, -0.2) (0.2, 1)};
			\end{tikzpicture}
			\end{center}
		\end{column}
		\begin{column}{.3\textwidth}
			\begin{center}
			\begin{tikzpicture}
				\node [bnode, label=below:{$a$}] (a) at (-1, 0) {};
				\node [bnode, label={$\downarrow \! b = \{a, b\}$}] (b) at (-2, 1) {};
				\node [wnode, label=right:{$c$}] (c) at (0, 1) {};
				\path [line] (a) -- (b);
				\path [line] (a) -- (c);
				\draw [very thick, dotted] plot [smooth, tension=0.5] coordinates {(-2.2, 0) (-2, 1.2) (-0.8, 0)};
			\end{tikzpicture}
			\end{center}
		\end{column}
	\end{columns}
\end{frame}

\subsection{Treillis}

\begin{frame}
	\frametitle{Treillis}
	\begin{columns}[T]
		\begin{column}{.6\textwidth}
			\begin{definition}
				Soit un treillis $(T, \leq, \vee, \wedge)$ :
				\begin{itemize}
					\item $\forall x, y \in T$, $\exists z \in T : x \vee y = z$
					\item $\forall x, y \in T$, $\exists z \in T : x \wedge y = z$
				\end{itemize}
			\end{definition}
			\begin{itemize}
				\item Ensemble ordonné
				\item Opération de borne supérieur $\vee$
				\item Opération de borne inférieur $\wedge$
			\end{itemize}
		\end{column}
		\begin{column}{.3\textwidth}
			\begin{tikzpicture}
				\node [wnode, label=below:{$a = b \wedge c$}] (a) at (0, 0) {};
				\node [wnode, label=left:{$b$}] (b) at (-1, 1) {};
				\node [wnode, label=right:{$c$}] (c) at (1, 1) {};
				\node [wnode, label={$d = b \vee c$}] (d) at (0, 2) {};
				\path [line] (a) -- (b);
				\path [line] (a) -- (c);
				\path [line] (b) -- (d);
				\path [line] (c) -- (d);
			\end{tikzpicture}
		\end{column}
	\end{columns}
\end{frame}

\begin{frame}
	\frametitle{Irréductibles}
	\begin{columns}[T]
		\begin{column}{.5\textwidth}
			\begin{definition}
				Soit un treillis $(T, \leq, \vee, \wedge)$, $x, y, z \in T$ :
				\begin{itemize}
					\item $x$ est $\vee$-irreductible ssi $x \vee y = z \Rightarrow x = z$
					\item $x$ est $\wedge$-irreductible ssi $x \wedge y = z \Rightarrow x = z$
				\end{itemize}
			\end{definition}
		\end{column}
		\begin{column}{.5\textwidth}
			\begin{tikzpicture}
				\node [wnode, label=below:{$a$}] (a) at (0, 0) {};
				\node [wnode, label=left:{$b = b \wedge d$}] (b) at (-1, 1) {};
				\node [wnode, label=right:{$c = a \vee c$}] (c) at (1, 1) {};
				\node [wnode, label={$d$}] (d) at (0, 2) {};
				\path [line] (a) -- (b);
				\path [line] (a) -- (c);
				\path [line] (b) -- (d);
				\path [line] (c) -- (d);
			\end{tikzpicture}
		\end{column}
	\end{columns}
\end{frame}

\begin{frame}
	\frametitle{Treillis distributif}
	\only<1| handout:1> {
		\begin{definition}
			Un treillis $(T, \leq, \vee, \wedge)$ est distributif si et seulement si au moins l'une des trois conditions équivalentes suivantes est vérifiée :
			\begin{itemize}
				\item $(x \wedge y) \vee (x \wedge z) \vee (y \wedge z) = (x \vee y) \wedge (x \vee z) \wedge (y \vee z)$
				\item Le treillis ne possède ni $N_5$ ni $M_3$ en guise de sous treillis
				\item Le contexte réduit contient une seule relation flèche double par ligne et par colonne
			\end{itemize}
		\end{definition}
	}\only<2| handout:2> {
		\begin{columns}
			\begin{column}{.4\textwidth}
				\begin{tikzpicture}
					\node [wnode, label=below:{$\bot$}] (bot) at (0,0) {};
					\node [bnode, label=left:{1, $a$}] (1) at (-1, 1) {};
					\node [bnode, label=right:{2, $b$}] (2) at (1, 1) {};
					\node [bnode, label=left:{3, $c$}] (3) at (-1, 2) {};
					\node [wnode, label={$\top$}] (top) at (0, 3) {};
					\path [line] (bot) -- (1);
					\path [line] (bot) -- (2);
					\path [line] (1) -- (3);
					\path [line] (2) -- (top);
					\path [line] (3) -- (top);
				\end{tikzpicture}
			\end{column}
			\begin{column}{.4\textwidth}
				\begin{tikzpicture}
					\node [wnode, label=below:{$\bot$}] (bot) at (0,0) {};
					\node [bnode, label=right:{1, $a$}] (1) at (-1.5, 1.5) {};
					\node [bnode, label=right:{2, $b$}] (2) at (0, 1.5) {};
					\node [bnode, label=right:{3, $c$}] (3) at (1.5, 1.5) {};
					\node [wnode, label={$\top$}] (top) at (0, 3) {};
					\path [line] (bot) -- (1);
					\path [line] (bot) -- (2);
					\path [line] (bot) -- (3);
					\path [line] (1) -- (top);
					\path [line] (2) -- (top);
					\path [line] (3) -- (top);
				\end{tikzpicture}
			\end{column}
		\end{columns}
	}
\end{frame}

\subsection{Graphe médian}

\begin{frame}
	\frametitle{Graphe médian}
	\begin{itemize}
		\item Un unique n\oe ud commun à tous les chemins les plus courts entre les éléments d'un triplets de n\oe uds
		\item Liens forts avec les treillis distributifs
	\end{itemize}
\end{frame}

\begin{frame}
	\frametitle{Graphe médian}
	\begin{columns}
		\begin{column}{.4\textwidth}
			\begin{tikzpicture}
				\node [wnode, label=below:{$\bot$}] (bot) at (0,0) {};
				\node [wnode, label=left:{1, $a$}] (1) at (-1, 1) {};
				\node [wnode, label=right:{2, $b$}] (2) at (1, 1) {};
				\node [wnode, label=left:{3, $c$}] (3) at (-1, 2) {};
				\node [wnode, label={$\top$}] (top) at (0, 3) {};
				\path [line] (bot) -- (1);
				\path [line] (bot) -- (2);
				\path [line] (1) -- (3);
				\path [line] (2) -- (top);
				\path [line] (3) -- (top);
			\end{tikzpicture}
		\end{column}
		\begin{column}{.4\textwidth}
			\begin{tikzpicture}
				\node [wnode, label=below:{$\bot$}] (bot) at (0,0) {};
				\node [wnode, label=right:{1, $a$}] (1) at (-1.5, 1.5) {};
				\node [wnode, label=right:{2, $b$}] (2) at (0, 1.5) {};
				\node [wnode, label=right:{3, $c$}] (3) at (1.5, 1.5) {};
				\node [wnode, label={$\top$}] (top) at (0, 3) {};
				\path [line] (bot) -- (1);
				\path [line] (bot) -- (2);
				\path [line] (bot) -- (3);
				\path [line] (1) -- (top);
				\path [line] (2) -- (top);
				\path [line] (3) -- (top);
			\end{tikzpicture}
		\end{column}
		\begin{column}{.4\textwidth}
			\begin{tikzpicture}
				\node [wnode, label=below:{$\bot$}] (bot) at (0,0) {};
				\node [wnode, label=left:{1}] (1) at (-1, 1) {};
				\node [wnode, label=right:{2, $b$}] (2) at (1, 1) {};
				\node [wnode, label=left:{3, $c$}] (3) at (-1, 2) {};
				\node [wnode, label=right:{$a$}] (A) at (1, 2) {};
				\node [wnode, label={$\top$}] (top) at (0, 3) {};
				\path [line] (bot) -- (1);
				\path [line] (bot) -- (2);
				\path [line] (1) -- (3);
				\path [line] (1) -- (A);
				\path [line] (2) -- (A);
				\path [line] (3) -- (top);
				\path [line] (A) -- (top);
			\end{tikzpicture}
		\end{column}
	\end{columns}
\end{frame}

\section{Existant}

\subsection{Méthode}

\begin{frame}
	\frametitle{Méthode}
	\begin{itemize}
		\item Treillis distributif $\Rightarrow$ graphe médian
		\item Plus proche possible du treillis d'origine
	\end{itemize}
	\begin{itemize}
		\item Rendre distributifs les treillis des filtres des atomes
	\end{itemize}
\end{frame}

\begin{frame}
	\frametitle{Méthode}
	\vspace{-1cm} 
	\begin{columns}[c]
		\begin{column}{.3\textwidth}
	\begin{center}
		\begin{tikzpicture}
			\node [wnode, label=below:{$\bot$}] (bot) at (-1, -1) {};
			\node [wnode, label=left:{$1$}] (1) at (-2, 0) {};
			\node [wnode, label=right:{$2$}] (2) at (0, 0) {};
			\node [wnode, label=left:{$3$}] (3) at (-1, 1) {};
			\node [wnode, label=right:{$4$}] (4) at (1, 1) {};
			\node [wnode, label=right:{$5$}] (5) at (1, 2) {};
			\node [wnode, label=right:{$6$}] (6) at (0, 3) {};
			\node [wnode, label={$\top$}] (top) at (-1, 4) {};
			
			\path [line] (bot) -- (1);
			\path [line] (bot) -- (2);
			\path [line] (1) -- (top);
			\path [line] (2) -- (3);
			\path [line] (2) -- (4);
			\path [line] (3) -- (6);
			\path [line] (4) -- (5);
			\path [line] (5) -- (6);
			\path [line] (6) -- (top);
		\end{tikzpicture}
	\end{center}
		\end{column}
		\begin{column}{.3\textwidth}
	\begin{center}
		\begin{tikzpicture}
			\node [wnode, label=below:{$\bot$}] (bot) at (-1, -1) {};
			\node [wnode, label=left:{$1$}] (1) at (-2, 0) {};
			\node [wnode, label=right:{$2$}] (2) at (0, 0) {};
			\node [wnode, label=left:{$3$}] (3) at (-1, 1) {};
			\node [wnode, label=right:{$4$}] (4) at (1, 1) {};
			\node [wnode, label=right:{$5$}] (5) at (1, 2) {};
			\node [wnode, label=right:{$6$}] (6) at (0, 3) {};
			\node [wnode, label=left:{$7$}] (7) at (-1, 2) {};
			\node [wnode, label={$\top$}] (top) at (-1, 4) {};
			
			\path [line] (bot) -- (1);
			\path [line] (bot) -- (2);
			\path [line] (1) -- (top);
			\path [line] (2) -- (3);
			\path [line] (2) -- (4);
			\path [line] (3) -- (7);
			\path [line] (4) -- (5);
			\path [line] (4) -- (7);
			\path [line] (5) -- (6);
			\path [line] (6) -- (top);
			\path [line] (7) -- (6);
		\end{tikzpicture}
	\end{center}
		\end{column}
		\begin{column}{.4\textwidth}
	\begin{center}
		\begin{tikzpicture}
			\node [wnode, label=left:{$1$}] (1) at (-2, 0) {};
			\node [wnode, label=right:{$2$}] (2) at (0, 0) {};
			\node [wnode, label=left:{$3$}] (3) at (-1, 1) {};
			\node [wnode, label=right:{$4$}] (4) at (1, 1) {};
			\node [wnode, label=right:{$5$}] (5) at (1, 2) {};
			\node [wnode, label=right:{$6$}] (6) at (0, 3) {};
			\node [wnode, label=left:{$7$}] (7) at (-1, 2) {};
			\node [wnode, label={$\top$}] (top) at (-1, 4) {};
			
			\path [line] (1) -- (top);
			\path [line] (2) -- (3);
			\path [line] (2) -- (4);
			\path [line] (3) -- (7);
			\path [line] (4) -- (5);
			\path [line] (4) -- (7);
			\path [line] (5) -- (6);
			\path [line] (6) -- (top);
			\path [line] (7) -- (6);
		\end{tikzpicture}
	\end{center}
		\end{column}
	\end{columns}
\end{frame}

%\subsection{Extraction}

%\begin{frame}
%	\frametitle{Extraction}
%\end{frame}

\subsection{Algorithme}

\begin{frame}
	\frametitle{Algorithme}
	\begin{columns}
		\begin{column}{.4\textwidth}
			\begin{center}
			\begin{tabular}{ l | c c c }
				 & A & B & C \\
				\hline
				1 & x & & x \\
				2 & & x & \\
				3 & & & x \\
			\end{tabular}
			\end{center}
		\end{column}
		\begin{column}{.4\textwidth}
			\begin{center}
			\begin{tikzpicture}
				\node [wnode, label=below:{$\bot$}] (bot) at (0,0) {};
				\node [bnode, label=left:{1, A}] (1) at (-1, 1) {};
				\node [bnode, label=right:{2, B}] (2) at (1, 1) {};
				\node [bnode, label=left:{3, C}] (3) at (-1, 2) {};
				\node [wnode, label={$\top$}] (top) at (0, 3) {};
				\path [line] (bot) -- (1);
				\path [line] (bot) -- (2);
				\path [line] (1) -- (3);
				\path [line] (2) -- (top);
				\path [line] (3) -- (top);
			\end{tikzpicture}
			\end{center}
		\end{column}
	\end{columns}
\end{frame}

\begin{frame}
	\frametitle{Algorithme}
	\begin{columns}
		\begin{column}{.4\textwidth}
			\begin{center}
			\begin{tabular}{ l | c }
				 & m1\\
				\hline
				1 & \\
				2 & x \\
				3 & \\
			\end{tabular}
			\end{center}
		\end{column}
		\begin{column}{.4\textwidth}
			\begin{center}
			\begin{tikzpicture}
				\node [wnode, label=below:{$\bot$}] (bot) at (0,0) {};
				\node [bnode, label=left:{1}] (1) at (-1, 1) {};
				\node [bnode, label=right:{2}] (2) at (1, 1) {};
				\node [bnode, label=left:{3}] (3) at (-1, 2) {};
				\node [wnode, label={$\top$}] (top) at (0, 3) {};
				\path [line] (bot) -- (1);
				\path [line] (bot) -- (2);
				\path [line] (1) -- (3);
				\path [line] (2) -- (top);
				\path [line] (3) -- (top);	
				\draw [very thick, dotted] plot [smooth, tension=2] coordinates {(-2.5, 3.5) (-1, 0.5) (0.5, 3.5)};
			\end{tikzpicture}
			\end{center}
		\end{column}
	\end{columns}
\end{frame}

\begin{frame}
	\frametitle{Algorithme}
	\begin{columns}
		\begin{column}{.4\textwidth}
			\begin{center}
			\begin{tabular}{ l | c c }
				 & m1 & m2\\
				\hline
				1 & & x\\
				2 & x & \\
				3 & & x \\
			\end{tabular}
			\end{center}
		\end{column}
		\begin{column}{.4\textwidth}
			\begin{center}
			\begin{tikzpicture}
				\node [wnode, label=below:{$\bot$}] (bot) at (0,0) {};
				\node [bnode, label=left:{1}] (1) at (-1, 1) {};
				\node [bnode, label=right:{2}] (2) at (1, 1) {};
				\node [bnode, label=left:{3}] (3) at (-1, 2) {};
				\node [wnode, label={$\top$}] (top) at (0, 3) {};
				
				\path [line] (bot) -- (1);
				\path [line] (bot) -- (2);
				\path [line] (1) -- (3);
				\path [line] (2) -- (top);
				\path [line] (3) -- (top);
				
				\draw [very thick, dotted] plot [smooth, tension=2] coordinates {(-0.5, 3.5) (1, 0.5) (2.5, 3.5)};
			\end{tikzpicture}
			\end{center}
		\end{column}
	\end{columns}
\end{frame}

\begin{frame}
	\frametitle{Algorithme}
	\begin{columns}
		\begin{column}{.4\textwidth}
			\begin{center}
			\begin{tabular}{ l | c c c }
				 & m1 & m2 & m3 \\
				\hline
				1 & & x & x \\
				2 & x & & x\\
				3 & & x & \\
			\end{tabular}
			\end{center}
		\end{column}
		\begin{column}{.4\textwidth}
			\begin{center}
			\begin{tikzpicture}
				\node [wnode, label=below:{$\bot$}] (bot) at (0,0) {};
				\node [bnode, label=left:{1}] (1) at (-1, 1) {};
				\node [bnode, label=right:{2}] (2) at (1, 1) {};
				\node [bnode, label=left:{3}] (3) at (-1, 2) {};
				\node [wnode, label={$\top$}] (top) at (0, 3) {};
				
				\path [line] (bot) -- (1);
				\path [line] (bot) -- (2);
				\path [line] (1) -- (3);
				\path [line] (2) -- (top);
				\path [line] (3) -- (top);
				
				\draw [very thick, dotted] plot [smooth, tension=2] coordinates {(-2.5, 3.5) (-1, 1.5) (0.5, 3.5)};
			\end{tikzpicture}
			\end{center}
		\end{column}
	\end{columns}
\end{frame}

\begin{frame}
	\frametitle{Algorithme}
	\begin{columns}
		\begin{column}{.4\textwidth}
			\begin{center}
			\begin{tabular}{ l | c c c }
				 & m1 & m2 & m3 \\
				\hline
				1 & & x & x \\
				2 & x & & x\\
				3 & & x & \\
			\end{tabular}
			\end{center}
		\end{column}
		\begin{column}{.4\textwidth}
			\begin{center}
			\begin{tikzpicture}
				\node [wnode, label=below:{$\bot$}] (bot) at (0,0) {};
				\node [wnode, label=left:{1}] (1) at (-1, 1) {};
				\node [wnode, label=right:{2, m1}] (2) at (1, 1) {};
				\node [wnode, label=left:{3, m2}] (3) at (-1, 2) {};
				\node [bnode, label=right:{m3}] (4) at (1, 2) {};
				\node [wnode, label={$\top$}] (top) at (0, 3) {};
				
				\path [line] (bot) -- (1);
				\path [line] (bot) -- (2);
				\path [line] (1) -- (3);
				\path [line] (1) -- (4);
				\path [line] (2) -- (4);
				\path [line] (3) -- (top);
				\path [line] (4) -- (top);
			\end{tikzpicture}
			\end{center}
		\end{column}
	\end{columns}
\end{frame}

%\subsection{Remise en commun}

%\begin{frame}
%	\frametitle{Remise en commun}
%\end{frame}

\subsection{Problèmes}

\begin{frame}
	\frametitle{Problèmes}
	\begin{center}
	\begin{tikzpicture}
		\node [wnode, label=below:{$\bot$}] (bot) at (0,0) {};
		\node [bnode, label=left:{1}] (1) at (-1, 1) {};
		\node [wnode, label=right:{4}] (4) at (1, 1) {};
		\node [bnode, label=left:{2, $A$}] (2) at (-2, 2) {};
		\node [bnode, label=left:{3, $B$}] (3) at (-2, 3) {};
		\node [wnode, label=right:{5, $D$}] (5) at (2, 2) {};
		\node [wnode, label=right:{6, $E$}] (6) at (2, 3) {};
		\node [bnode, label=right:{$C$}] (C) at (0, 2) {};
		\node [bnode, label={$\top$}] (top) at (0, 4) {};
		\path [line] (bot) -- (1);
		\path [line] (bot) -- (4);
		\path [line] (1) -- (2);
		\path [line] (1) -- (C);
		\path [line] (2) -- (3);
		\path [line] (3) -- (top);
		\path [line] (4) -- (5);
		\path [line] (4) -- (C);
		\path [line] (5) -- (6);
		\path [line] (6) -- (top);
		\path [line] (C) -- (top);
	\end{tikzpicture}
	\end{center}
\end{frame}

\begin{frame}
	\frametitle{Problèmes}
	\begin{center}
	\begin{tikzpicture}
		\node [wnode, label=below:{$\bot$}] (bot) at (0,0) {};
		\node [bnode, label=left:{1}] (1) at (-1, 1) {};
		\node [wnode, label=right:{4}] (4) at (1, 1) {};
		\node [bnode, label=left:{2}] (2) at (-2, 2) {};
		\node [bnode, label=left:{3, $m4$}] (3) at (-2, 3) {};
		\node [wnode, label=right:{5}] (5) at (2, 2) {};
		\node [wnode, label=right:{6, $m1$}] (6) at (2, 3) {};
		\node [wnode, label=right:{$m2$, $m5$}] (m2) at (0, 2) {};
		\node [wnode, label=left:{$m3$}] (m3) at (-0.5, 3) {};
		\node [bnode, label=right:{$m6$}] (m6) at (0.5, 3) {};
		\node [bnode, label={$\top$}] (top) at (0, 4) {};
		\path [line] (bot) -- (1);
		\path [line] (bot) -- (4);
		\path [line] (1) -- (2);
		\path [line] (1) -- (m2);
		\path [line] (2) -- (3);
		\path [line] (3) -- (top);
		\path [line] (4) -- (5);
		\path [line] (4) -- (m2);
		\path [line] (5) -- (6);
		\path [line] (6) -- (top);
		\path [line] (m2) -- (m3);
		\path [line] (m2) -- (m6);
		\path [line] (m3) -- (top);
		\path [line] (m6) -- (top);
		\path [line] (2) -- (m3);
		\path [line] (5) -- (m6);
	\end{tikzpicture}
	\end{center}
\end{frame}

\begin{frame}
	\frametitle{Problèmes}
	\begin{center}
	\scalebox{0.9}{
	\begin{tikzpicture}
		\node [bnode, label=below:{$\bot$}] (bot) at (0,0) {};
		\node [bnode, label=left:{1}] (1) at (-1, 1) {};
		\node [bnode, label=right:{4}] (4) at (1, 1) {};
		\node [bnode, label=left:{2}] (2) at (-2, 2) {};
		\node [bnode, label=left:{3}] (3) at (-3, 3) {};
		\node [bnode, label=right:{5}] (5) at (2, 2) {};
		\node [bnode, label=right:{6}] (6) at (3, 3) {};
		\node [bnode, label=right:{}] (14) at (0, 2) {};
		\node [bnode, label={$\top$}] (top) at (0, 6) {};
		\node [wnode, label=left:{$m3$}] (7) at (-1, 3) {};
		\node [wnode, label=right:{$m6$}] (8) at (1, 3) {};
		\node [wnode, label=left:{}] (9) at (-2, 4) {};
		\node [wnode, label=right:{}] (10) at (2, 4) {};
		\node [wnode, label={}] (11) at (0, 4) {};
		\node [wnode, label=left:{}] (12) at (-1, 5) {};
		\node [wnode, label=right:{}] (13) at (1, 5) {};
		\path [line] (bot) -- (1);
		\path [line] (bot) -- (4);
		\path [line] (1) -- (2);
		\path [line] (1) -- (C);
		\path [line] (2) -- (3);
		\path [line] (4) -- (5);
		\path [line] (4) -- (C);
		\path [line] (5) -- (6);
		\path [line] (2) -- (7);
		\path [line] (5) -- (8);
		\path [line] (14) -- (7);
		\path [line] (14) -- (8);
		\path [line] (3) -- (9);
		\path [line] (9) -- (12);
		\path [line] (12) -- (top);
		\path [line] (7) -- (11);
		\path [line] (11) -- (13);
		\path [line] (8) -- (10);
		\path [line] (7) -- (9);
		\path [line] (8) -- (11);
		\path [line] (11) -- (12);
		\path [line] (6) -- (10);
		\path [line] (10) -- (13);
		\path [line] (13) -- (top);
	\end{tikzpicture}
	}
	\end{center}
\end{frame}

\begin{frame}
	\frametitle{Problèmes}
	\begin{center}
	\begin{tikzpicture}
		\node [bnode, label=below:{bot}] (bot) at (0,0) {};
		\node [bnode, label=left:{1}] (1) at (-1, 1) {};
		\node [bnode, label=right:{4}] (4) at (1, 1) {};
		\node [bnode, label=left:{2}] (2) at (-2, 2) {};
		\node [bnode, label=left:{3, m4}] (3) at (-2, 3) {};
		\node [bnode, label=right:{5}] (5) at (2, 2) {};
		\node [bnode, label=right:{6, m1}] (6) at (2, 3) {};
		\node [bnode, label=right:{m2, m5}] (m2) at (0, 2) {};
		\node [wnode, label=right:{m3, m6}] (m3) at (0, 3) {};
		\node [bnode, label={top}] (top) at (0, 4) {};
		\path [line] (bot) -- (1);
		\path [line] (bot) -- (4);
		\path [line] (1) -- (2);
		\path [line] (1) -- (m2);
		\path [line] (2) -- (3);
		\path [line] (3) -- (top);
		\path [line] (4) -- (5);
		\path [line] (4) -- (m2);
		\path [line] (5) -- (6);
		\path [line] (6) -- (top);
		\path [line] (m2) -- (m3);
		\path [line] (m3) -- (top);
		\path [line] (2) -- (m3);
		\path [line] (5) -- (m3);
	\end{tikzpicture}
	\end{center}
\end{frame}

\section{Contribution}

\subsection{Implémentation}

\begin{frame}
	\frametitle{Implémentation : workflow}
	\begin{center}
	\begin{tikzpicture}[node distance = 0.5cm, auto]
		% Place nodes
		\node [inout] (start) at (0, 0) {début};
		\node [inout] (end) at (3, 0) {fin};
		\node [block] (import) at (0, -1.5) {importation};
		\node [block] (export) at (3, -1.5) {exportation};
		\node [algorithm] (eachatom) at (0, -3) {pour chaque atome};
		\node [block] (extract) at (0, -4.5) {extraction du contexte};
		\node [block] (algo) at (3, -4.5) {algorithme};
		\node [block] (commun) at (6, -4.5) {remise en commun};
		% Draw edges
		\path [line2] (start) -- (import);
		\path [line2] (import) -- (eachatom);
		\path [line2] (export) -- (end);
		\path [line3] (eachatom) -- (extract);
		\path [line2] (eachatom) -| (export);
		\path [line3] (extract) -- (algo);
		\path [line3] (algo) -- (commun);
	\end{tikzpicture}
	\end{center}
\end{frame}

\subsection{Fusion}

\begin{frame}
	\frametitle{Fusion}
	\begin{columns}
		\begin{column}{.4\textwidth}
			Les deux n\oe uds à fusionner ne doivent pas faire partis des n\oe uds du treillis de départ.
		\end{column}
		\begin{column}{.55\textwidth}
			\begin{center}
			\begin{tikzpicture}
				\node [bnode, label=below:{$\bot$}] (bot) at (0,0) {};
				\node [bnode, label=left:{1}] (1) at (-1, 1) {};
				\node [bnode, label=right:{4}] (4) at (1, 1) {};
				\node [bnode, label=left:{2}] (2) at (-2, 2) {};
				\node [bnode, label=left:{3, $m4$}] (3) at (-2, 3) {};
				\node [bnode, label=right:{5}] (5) at (2, 2) {};
				\node [bnode, label=right:{6, $m1$}] (6) at (2, 3) {};
				\node [bnode, label=right:{$m2$, $m5$}] (m2) at (0, 2) {};
				\node [wnode, label=left:{$m3$}] (m3) at (-0.5, 3) {};
				\node [wnode, label=right:{$m6$}] (m6) at (0.5, 3) {};
				\node [bnode, label={$\top$}] (top) at (0, 4) {};
				\path [line] (bot) -- (1);
				\path [line] (bot) -- (4);
				\path [line] (1) -- (2);
				\path [line] (1) -- (m2);
				\path [line] (2) -- (3);
				\path [line] (3) -- (top);
				\path [line] (4) -- (5);
				\path [line] (4) -- (m2);
				\path [line] (5) -- (6);
				\path [line] (6) -- (top);
				\path [line] (m2) -- (m3);
				\path [line] (m2) -- (m6);
				\path [line] (m3) -- (top);
				\path [line] (m6) -- (top);
				\path [line] (2) -- (m3);
				\path [line] (5) -- (m6);
			\end{tikzpicture}
			\end{center}
		\end{column}
	\end{columns}
\end{frame}

\begin{frame}
	\frametitle{Fusion}
	\begin{columns}
		\begin{column}{.4\textwidth}
			Les deux n\oe uds à fusionner doivent être en couverture d'un n\oe ud présent dans au moins deux filtres d'atomes.
		\end{column}
		\begin{column}{.55\textwidth}
			\begin{center}
			\begin{tikzpicture}
				\node [wnode, label=below:{$\bot$}] (bot) at (0,0) {};
				\node [wnode, label=left:{1}] (1) at (-1, 1) {};
				\node [wnode, label=right:{4}] (4) at (1, 1) {};
				\node [wnode, label=left:{2}] (2) at (-2, 2) {};
				\node [wnode, label=left:{3, $m4$}] (3) at (-2, 3) {};
				\node [wnode, label=right:{5}] (5) at (2, 2) {};
				\node [wnode, label=right:{6, $m1$}] (6) at (2, 3) {};
				\node [bnode, label=right:{$m2$, $m5$}] (m2) at (0, 2) {};
				\node [wnode, label=left:{$m3$}] (m3) at (-0.5, 3) {};
				\node [wnode, label=right:{$m6$}] (m6) at (0.5, 3) {};
				\node [wnode, label={$\top$}] (top) at (0, 4) {};
				\path [line] (bot) -- (1);
				\path [line] (bot) -- (4);
				\path [line] (1) -- (2);
				\path [line] (1) -- (m2);
				\path [line] (2) -- (3);
				\path [line] (3) -- (top);
				\path [line] (4) -- (5);
				\path [line] (4) -- (m2);
				\path [line] (5) -- (6);
				\path [line] (6) -- (top);
				\path [line] (m2) -- (m3);
				\path [line] (m2) -- (m6);
				\path [line] (m3) -- (top);
				\path [line] (m6) -- (top);
				\path [line] (2) -- (m3);
				\path [line] (5) -- (m6);
			\end{tikzpicture}
			\end{center}
		\end{column}
	\end{columns}
\end{frame}

\begin{frame}
	\frametitle{Fusion}
	\begin{columns}
		\begin{column}{.4\textwidth}
			Les deux n\oe uds à fusionner ne doivent pas avoir de n\oe uds en commun dans leurs idéaux une fois que les idéaux du n\oe ud en commun (celui dont il est question dans la condition précédente) leur sont retirés.
		\end{column}
		\begin{column}{.55\textwidth}
			\begin{center}
			\begin{tikzpicture}
				\node [wnode, label=below:{$\bot$}] (bot) at (0,0) {};
				\node [wnode, label=left:{1}] (1) at (-1, 1) {};
				\node [wnode, label=right:{4}] (4) at (1, 1) {};
				\node [bnode, label=left:{2}] (2) at (-2, 2) {};
				\node [wnode, label=left:{3, $m4$}] (3) at (-2, 3) {};
				\node [bnode, label=right:{5}] (5) at (2, 2) {};
				\node [wnode, label=right:{6, $m1$}] (6) at (2, 3) {};
				\node [wnode, label=right:{$m2$, $m5$}] (m2) at (0, 2) {};
				\node [wnode, label=left:{$m3$}] (m3) at (-0.5, 3) {};
				\node [wnode, label=right:{$m6$}] (m6) at (0.5, 3) {};
				\node [wnode, label={$\top$}] (top) at (0, 4) {};
				\path [line] (bot) -- (1);
				\path [line] (bot) -- (4);
				\path [line] (1) -- (2);
				\path [line] (1) -- (m2);
				\path [line] (2) -- (3);
				\path [line] (3) -- (top);
				\path [line] (4) -- (5);
				\path [line] (4) -- (m2);
				\path [line] (5) -- (6);
				\path [line] (6) -- (top);
				\path [line] (m2) -- (m3);
				\path [line] (m2) -- (m6);
				\path [line] (m3) -- (top);
				\path [line] (m6) -- (top);
				\path [line] (2) -- (m3);
				\path [line] (5) -- (m6);
			\end{tikzpicture}
			\end{center}
		\end{column}
	\end{columns}
\end{frame}

\begin{frame}
	\frametitle{Fusion : workflow}
	\begin{center}
	\begin{tikzpicture}[node distance = 0.5cm, auto]
		% Place nodes
		\node [inout] (start) at (0, 0) {début};
		\node [inout] (end) at (3, 0) {fin};
		\node [block] (import) at (0, -1.5) {importation};
		\node [block] (export) at (3, -1.5) {exportation};
		\node [algorithm] (eachatom) at (0, -3) {pour chaque atome};
		\node [block] (extract) at (0, -4.5) {extraction du contexte};
		\node [block] (algo) at (3, -4.5) {algorithme};
		\node [block] (commun) at (6, -4.5) {remise en commun};
		\node [block] (fusion) at (3, -3) {fusion};
		% Draw edges
		\path [line2] (start) -- (import);
		\path [line2] (import) -- (eachatom);
		\path [line2] (export) -- (end);
		\path [line3] (eachatom) -- (extract);
		\path [line2] (eachatom) -- (fusion);
		\path [line3] (extract) -- (algo);
		\path [line3] (algo) -- (commun);
		\path [line2] (fusion) -- (export);
	\end{tikzpicture}
	\end{center}
\end{frame}

\subsection{Problèmes rencontrés}

\begin{frame}
	\frametitle{Système de boucle}
	\begin{center}
	\begin{tikzpicture}
		\node [wnode, label=below:{$\bot$}] (bot) at (0,0) {};
		\node [bnode, label=right:{1}] (1) at (1, 1) {};
		\node [wnode, label=left:{2}] (2) at (-1, 1) {};
		\node [bnode, label=right:{3}] (3) at (0, 2) {};
		\node [bnode, label=right:{4}] (4) at (2, 2) {};
		\node [wnode, label=left:{5}] (5) at (-2, 2) {};
		\node [bnode, label=right:{6}] (6) at (2, 3) {};
		\node [bnode, label={$\top$}] (top) at (0, 4) {};
	
		\path [line] (bot) -- (1);
		\path [line] (bot) -- (2);
		\path [line] (1) -- (3);
		\path [line] (1) -- (4);
		\path [line] (2) -- (3);
		\path [line] (2) -- (5);
		\path [line] (4) -- (6);
		\path [line] (6) -- (top);
		\path [line] (3) -- (top);
		\path [line] (5) -- (top);
	\end{tikzpicture}
	\end{center}
\end{frame}

\begin{frame}
	\frametitle{Système de boucle}
	\begin{columns}
		\begin{column}{.48\textwidth}
			\begin{center}
			\begin{tikzpicture}
				\node [wnode, label=below:{$\bot$}] (bot) at (0,0) {};
				\node [bnode, label=right:{1}] (1) at (1, 1) {};
				\node [wnode, label=left:{2}] (2) at (-1, 1) {};
				\node [bnode, label=right:{3}] (3) at (0, 2) {};
				\node [bnode, label=right:{4}] (4) at (2, 2) {};
				\node [wnode, label=left:{5}] (5) at (-2, 2) {};
				\node [bnode, label=right:{6}] (6) at (2, 3) {};
				\node [bnode, label={$\top$}] (top) at (0, 4) {};
			
				\path [line] (bot) -- (1);
				\path [line] (bot) -- (2);
				\path [line] (1) -- (3);
				\path [line] (1) -- (4);
				\path [line] (2) -- (3);
				\path [line] (2) -- (5);
				\path [line] (4) -- (6);
				\path [line] (6) -- (top);
				\path [line] (3) -- (top);
				\path [line] (5) -- (top);
			\end{tikzpicture}
			\end{center}
		\end{column}
		\begin{column}{.48\textwidth}
			\begin{center}
			\begin{tikzpicture}
				\node [wnode, label=below:{$\bot$}] (bot) at (0,0) {};
				\node [wnode, label=right:{1}] (1) at (1, 1) {};
				\node [bnode, label=left:{2}] (2) at (-1, 1) {};
				\node [bnode, label=right:{3}] (3) at (0, 2) {};
				\node [wnode, label=right:{4}] (4) at (2, 2) {};
				\node [bnode, label=left:{5}] (5) at (-2, 2) {};
				\node [wnode, label=right:{6}] (6) at (2, 3) {};
				\node [bnode, label=right:{}] (7) at (0, 3) {};
				\node [bnode, label={$\top$}] (top) at (0, 4) {};
			
				\path [line] (bot) -- (1);
				\path [line] (bot) -- (2);
				\path [line] (1) -- (3);
				\path [line] (1) -- (4);
				\path [line] (2) -- (3);
				\path [line] (2) -- (5);
				\path [line] (4) -- (6);
				\path [line] (4) -- (7);
				\path [line] (6) -- (top);
				\path [line] (3) -- (7);
				\path [line] (7) -- (top);
				\path [line] (5) -- (top);
			\end{tikzpicture}
			\end{center}
		\end{column}
	\end{columns}
\end{frame}

\begin{frame}
	\frametitle{Système de boucle}
	\begin{columns}
		\begin{column}{.48\textwidth}
			\begin{center}
			\begin{tikzpicture}
				\node [wnode, label=below:{$\bot$}] (bot) at (0,0) {};
				\node [wnode, label=right:{1}] (1) at (1, 1) {};
				\node [wnode, label=left:{2}] (2) at (-1, 1) {};
				\node [wnode, label=right:{3}] (3) at (0, 2) {};
				\node [wnode, label=right:{4}] (4) at (2, 2) {};
				\node [wnode, label=left:{5}] (5) at (-2, 2) {};
				\node [wnode, label=right:{6}] (6) at (2, 3) {};
				\node [bnode, label=right:{}] (7) at (0, 3) {};
				\node [wnode, label={$\top$}] (top) at (0, 4) {};
				\node [bnode, label=left:{}] (8) at (-2, 3) {};
				\path [line] (bot) -- (1);
				\path [line] (bot) -- (2);
				\path [line] (1) -- (3);
				\path [line] (1) -- (4);
				\path [line] (2) -- (3);
				\path [line] (2) -- (5);
				\path [line] (4) -- (6);
				\path [line] (4) -- (7);
				\path [line] (6) -- (top);
				\path [line] (3) -- (7);
				\path [line] (7) -- (top);
				\path [line] (5) -- (8);
				\path [line] (3) -- (8);
				\path [line] (8) -- (top);
			\end{tikzpicture}
			\end{center}
		\end{column}
		\begin{column}{.48\textwidth}
			\begin{center}
			\begin{tikzpicture}
				\node [wnode, label=below:{$\bot$}] (bot) at (0,0) {};
				\node [wnode, label=right:{1}] (1) at (1, 1) {};
				\node [wnode, label=left:{2}] (2) at (-1, 1) {};
				\node [wnode, label=right:{3}] (3) at (0, 2) {};
				\node [wnode, label=right:{4}] (4) at (2, 2) {};
				\node [wnode, label=left:{5}] (5) at (-2, 2) {};
				\node [wnode, label=right:{6}] (6) at (2, 3) {};
				\node [bnode, label=right:{}] (7) at (0, 3) {};
				\node [wnode, label={$\top$}] (top) at (0, 4) {};
				\path [line] (bot) -- (1);
				\path [line] (bot) -- (2);
				\path [line] (1) -- (3);
				\path [line] (1) -- (4);
				\path [line] (2) -- (3);
				\path [line] (2) -- (5);
				\path [line] (4) -- (6);
				\path [line] (4) -- (7);
				\path [line] (6) -- (top);
				\path [line] (3) -- (7);
				\path [line] (7) -- (top);
				\path [line] (5) -- (7);
			\end{tikzpicture}
			\end{center}
		\end{column}
	\end{columns}
\end{frame}

\begin{frame}
	\frametitle{Apparition de chaines}
	\begin{columns}
		\begin{column}{.48\textwidth}
			\begin{center}
			\begin{tikzpicture}
				\node [wnode, label=below:{$\bot$}] (bot) at (0,0) {};
				\node [wnode, label=left:{1}] (1) at (-0.5, 1) {};
				\node [wnode, label=right:{2}] (2) at (0.5, 1) {};
				\node [wnode, label=left:{3}] (3) at (-1, 2) {};
				\node [wnode, label=right:{}] (B) at (0, 2) {};
				\node [wnode, label=right:{4}] (4) at (1, 2) {};
				\node [wnode, label=left:{5}] (5) at (-1, 3) {};
				\node [wnode, label=right:{6}] (6) at (1, 3) {};
				\node [wnode, label=left:{7}] (7) at (-1, 4) {};
				\node [wnode, label=left:{}] (F) at (0, 4) {};
				\node [wnode, label={$\top$}] (top) at (0, 5) {};
				\path [line] (bot) -- (1);
				\path [line] (bot) -- (2);
				\path [line] (1) -- (3);
				\path [line] (1) -- (B);
				\path [line] (1) -- (3);
				\path [line] (2) -- (B);
				\path [line] (2) -- (4);
				\path [line] (3) -- (5);
				\path [line] (B) -- (F);
				\path [line] (4) -- (6);
				\path [line] (5) -- (7);
				\path [line] (5) -- (F);
				\path [line] (6) -- (top);
				\path [line] (7) -- (top);
				\path [line] (F) -- (top);
			\end{tikzpicture}
			\end{center}
		\end{column}
		\begin{column}{.48\textwidth}
			\begin{center}
			\begin{tikzpicture}
				\node [wnode, label=below:{$\bot$}] (bot) at (0,0) {};
				\node [wnode, label=left:{1}] (1) at (-0.5, 1) {};
				\node [wnode, label=right:{2}] (2) at (0.5, 1) {};
				\node [wnode, label=left:{3}] (3) at (-1, 2) {};
				\node [wnode, label=right:{}] (B) at (0, 2) {};
				\node [wnode, label=right:{4}] (4) at (1, 2) {};
				\node [wnode, label=left:{5}] (5) at (-1, 3) {};
				\node [wnode, label=right:{6}] (6) at (1, 3) {};
				\node [wnode, label=left:{7}] (7) at (-1, 4) {};
				\node [wnode, label=left:{}] (F) at (0, 4) {};
				\node [wnode, label=left:{}] (G) at (0, 3) {};
				\node [wnode, label={$\top$}] (top) at (0, 5) {};
			
				\path [line] (bot) -- (1);
				\path [line] (bot) -- (2);
				\path [line] (1) -- (3);
				\path [line] (1) -- (B);
				\path [line] (1) -- (3);
				\path [line] (2) -- (B);
				\path [line] (2) -- (4);
				\path [line] (3) -- (5);
				\path [line] (3) -- (G);
				\path [line] (B) -- (G);
				\path [line] (G) -- (F);
				\path [line] (4) -- (6);
				\path [line] (4) -- (G);
				\path [line] (5) -- (7);
				\path [line] (5) -- (F);
				\path [line] (6) -- (F);
				\path [line] (7) -- (top);
				\path [line] (F) -- (top);
			\end{tikzpicture}
			\end{center}
		\end{column}
	\end{columns}
\end{frame}

\begin{frame}
	\frametitle{Apparition de chaines}
	\begin{columns}
		\begin{column}{.48\textwidth}
			\begin{center}
			\scalebox{0.85}{
			\begin{tikzpicture}
				\node [wnode, label=below:{$\bot$}] (bot) at (0,0) {};
				\node [wnode, label=left:{1}] (1) at (-0.5, 1) {};
				\node [wnode, label=right:{2}] (2) at (0.5, 1) {};
				\node [wnode, label=left:{3}] (3) at (-1, 2) {};
				\node [wnode, label=right:{}] (B) at (0, 2) {};
				\node [wnode, label=right:{4}] (4) at (1, 2) {};
				\node [wnode, label=left:{5}] (5) at (-1.5, 3) {};
				\node [wnode, label={}] (x1) at (-0.5, 3) {};
				\node [wnode, label={}] (x2) at (0.5, 3) {};
				\node [wnode, label=right:{6}] (6) at (1.5, 3) {};
				\node [wnode, label={}] (x3) at (-0.5, 4) {};
				\node [wnode, label={}] (x4) at (1, 4) {};
				\node [wnode, label=left:{7}] (7) at (-1, 5) {};
				\node [wnode, label=left:{}] (F) at (0, 5) {};
				\node [wnode, label={$\top$}] (top) at (0, 6) {};
			
				\path [line] (bot) -- (1);
				\path [line] (bot) -- (2);
				\path [line] (1) -- (3);
				\path [line] (1) -- (B);
				\path [line] (1) -- (3);
				\path [line] (2) -- (B);
				\path [line] (2) -- (4);
				\path [line] (3) -- (5);
				\path [line] (3) -- (x1);
				\path [line] (B) -- (x1);
				\path [line] (B) -- (x2);
				\path [line] (4) -- (6);
				\path [line] (4) -- (x2);
				\path [line] (5) -- (7);
				\path [line] (5) -- (x3);
				\path [line] (x1) -- (x3);
				\path [line] (x2) -- (x4);
				\path [line] (x2) -- (F);
				\path [line] (6) -- (x4);
				\path [line] (x3) -- (F);
				\path [line] (x4) -- (top);
				\path [line] (7) -- (top);
				\path [line] (F) -- (top);
			\end{tikzpicture}
			}
			\end{center}
		\end{column}
		\begin{column}{.48\textwidth}
			\begin{center}
			\scalebox{0.85}{
			\begin{tikzpicture}
				\node [wnode, label=below:{$\bot$}] (bot) at (0,0) {};
				\node [wnode, label=left:{1}] (1) at (-0.5, 1) {};
				\node [wnode, label=right:{2}] (2) at (0.5, 1) {};
				\node [wnode, label=left:{3}] (3) at (-1, 2) {};
				\node [wnode, label=right:{}] (B) at (0, 2) {};
				\node [wnode, label=right:{4}] (4) at (1, 2) {};
				\node [wnode, label=left:{5}] (5) at (-1, 3) {};
				\node [wnode, label={}] (x1) at (0, 3) {};
				\node [wnode, label=right:{6}] (6) at (1, 3) {};
				\node [wnode, label={}] (x3) at (0, 4) {};
				\node [wnode, label=left:{7}] (7) at (-1, 5) {};
				\node [wnode, label=left:{}] (F) at (0, 5) {};
				\node [wnode, label={$\top$}] (top) at (0, 6) {};
			
				\path [line] (bot) -- (1);
				\path [line] (bot) -- (2);
				\path [line] (1) -- (3);
				\path [line] (1) -- (B);
				\path [line] (1) -- (3);
				\path [line] (2) -- (B);
				\path [line] (2) -- (4);
				\path [line] (3) -- (5);
				\path [line] (3) -- (x1);
				\path [line] (B) -- (x1);
				\path [line] (4) -- (6);
				\path [line] (4) -- (x1);
				\path [line] (5) -- (7);
				\path [line] (5) -- (x3);
				\path [line] (x1) -- (x3);
				\path [line] (6) -- (x3);
				\path [line] (x3) -- (F);
				\path [line] (7) -- (top);
				\path [line] (F) -- (top);
			\end{tikzpicture}
			}
			\end{center}
		\end{column}
	\end{columns}
\end{frame}

\begin{frame}
	\frametitle{Apparition de chaines}
	\begin{center}
	\scalebox{0.5}{
	\begin{tikzpicture}
		\node [wnode, label=below:{$\bot$}] (bot) at (0,0) {};
		\node [wnode, label=left:{1}] (1) at (-0.5, 1) {};
		\node [wnode, label=right:{2}] (2) at (0.5, 1) {};
		\node [wnode, label=left:{3}] (3) at (-1, 2) {};
		\node [wnode, label=right:{}] (B) at (0, 2) {};
		\node [wnode, label=right:{4}] (4) at (1, 2) {};
		\node [wnode, label=left:{5}] (5) at (-1, 3) {};
		\node [wnode, label={}] (x1) at (0, 3) {};
		\node [wnode, label=right:{6}] (6) at (1, 3) {};
		\node [wnode, label={}] (x3) at (0, 4) {};
		\node [wnode, label=left:{7}] (7) at (-1, 4) {};
		\node [wnode, label=left:{}] (F) at (-0.5, 5) {};
		\node [wnode, label={}] (x8) at (-0.5, 6) {};
		\node [wnode, label={}] (x10) at (0.5, 7) {};
		\node [wnode, label={}] (x11) at (-0.5, 7) {};
		\node [wnode, label={}] (x12) at (-1.5, 7) {};
		\node [wnode, label={}] (x13) at (0.5, 8) {};
		\node [wnode, label={}] (x14) at (-0.5, 8) {};
		\node [wnode, label={}] (x15) at (-1.5, 8) {};
		\node [wnode, label={}] (x16) at (0.5, 9) {};
		\node [wnode, label={}] (x17) at (-0.5, 9) {};
		\node [wnode, label={$\top$}] (top) at (0, 10) {};
	
		\path [line] (bot) -- (1);
		\path [line] (bot) -- (2);
		\path [line] (1) -- (3);
		\path [line] (1) -- (B);
		\path [line] (1) -- (3);
		\path [line] (2) -- (B);
		\path [line] (2) -- (4);
		\path [line] (3) -- (5);
		\path [line] (3) -- (x1);
		\path [line] (B) -- (x1);
		\path [line] (4) -- (6);
		\path [line] (4) -- (x1);
		\path [line] (5) -- (7);
		\path [line] (5) -- (x3);
		\path [line] (x1) -- (x3);
		\path [line] (6) -- (x3);
		\path [line] (x3) -- (F);
		\path [line] (7) -- (F);
		\path [line] (x3) -- (F);
		\path [line] (F) -- (x8);
		\path [line] (x8) -- (x10);
		\path [line] (x8) -- (x11);
		\path [line] (x8) -- (x12);
		\path [line] (x10) -- (x13);
		\path [line] (x10) -- (x14);
		\path [line] (x11) -- (x13);
		\path [line] (x11) -- (x15);
		\path [line] (x12) -- (x14);
		\path [line] (x12) -- (x15);
		\path [line] (x13) -- (x16);
		\path [line] (x13) -- (x17);
		\path [line] (x14) -- (x17);
		\path [line] (x15) -- (x17);
		\path [line] (x16) -- (top);
		\path [line] (x17) -- (top);
	\end{tikzpicture}
	}
	\end{center}
\end{frame}

\subsection{Ouverture}

\begin{frame}
	\frametitle{Système de boucle}
	\begin{itemize}
		\item Fusion au fur et à mesure
		\item Fusion après tout le système de boucle
	\end{itemize}
\end{frame}

\begin{frame}
	\frametitle{Apparition de chaines}
	\begin{itemize}
		\item Solution bancale
		\item Complication de la justification par preuves
		\item Recherche d'une solution plus adaptée
	\end{itemize}
\end{frame}

\begin{frame}
	\frametitle{Optimalité du treillis}
	\begin{itemize}
		\item Définition d'un système métrique de comparaison
		\item $D_{T1-T2}(a, b) : a$ est le nombre de n\oe uds ajoutés et $b$ le nombre d'arêtes ajoutées
		\item Quantifier la performance de la méthode
	\end{itemize}
\end{frame}

\begin{frame}
	\frametitle{Optimisation du programme}
	\begin{itemize}
		\item Gain de performance logicielle
		\item Simplification de certains morceaux de code
		\item Réduction des possibilités de bogues
	\end{itemize}
\end{frame}

\section{Bilan}

\subsection{Contribution}

\begin{frame}
	\frametitle{Contribution}
	\begin{itemize}
		\item Programme générant un treillis prêt à être converti en graphe médian pour la plupart des cas
		\item Proposition de trois conditions pour effectuer la fusion de n\oe uds
		\item Nécessitant d'y apporter des preuves
	\end{itemize}
\end{frame}

\subsection{Stage}

\begin{frame}
	\frametitle{Stage}
	\begin{itemize}
		\item Gains techniques classiques (Python, Latex, etc)
		\item Goût pour l'échange et le débat autour de notions
		\item Goût pour cette recherche constante d'amélioration même si on en n'est pas le bénéficiaire direct
	\end{itemize}
\end{frame}

\section*{}

%\subsection*{Remerciement}
%
%\begin{frame}
%\end{frame}

\subsection*{}

\begin{frame}
	\begin{center}
		% Choix du style de la biblio
		\bibliographystyle{unsrt}
		% Inclusion de la biblio
		\bibliography{bibliographie}
	\end{center}
\end{frame}

\begin{frame}
	\begin{center}
		Merci de votre attention.\\
		Avez-vous des question ?
	\end{center}
\end{frame}

\end{document}
