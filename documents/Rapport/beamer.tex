	\documentclass{beamer}

\usepackage[french]{babel}
\usepackage[T1]{fontenc}
\usepackage[utf8x]{inputenc}

\usetheme{Warsaw}
%\usetheme{Berkeley}
%\usetheme{Antibes}

\usepackage{subfig}
\usepackage{tikz}
\usetikzlibrary{shapes,arrows, positioning}

% tikz rules
\tikzstyle{bnode} = [circle, fill, draw, inner sep = 2pt]
\tikzstyle{wnode} = [circle, draw, inner sep = 2pt]
\tikzstyle{line} = [draw]

%\title[Graphes médians]{Production de graphes médians pour la phylogénie}
\title{Production de graphes médians pour la phylogénie}
\subtitle{Master 2 Sciences Cognitives et Médias Numériqes}
\author{Baptiste Mounier}
\institute{UFR Mathématiques et Informatique \and LORIA}
\date{09 avril 2018 - 08 août 2018}

%\usecolortheme{Magpie}

\AtBeginSection[]{
	\begin{frame}
		\frametitle{Sommaire}
		\tableofcontents[currentsection, hideothersubsections]
	\end{frame} 
}

\begin{document}

\begin{frame}[plain]
	\titlepage
\end{frame}

\begin{frame}{Sommaire}
	\tableofcontents[hideallsubsections]
\end{frame}


\section{Contexte}

\subsection{LORIA}

\begin{frame}
	\frametitle{LORIA}
	\begin{itemize}
		\item {\bf L}aboratoire L{\bf o}rrain de {\bf R}echerche en {\bf I}nforatique et ses {\bf A}pplications
		\item UMR 7503 : CNRS, Université de Lorraine et Inria
		\item Création en 1997
		\item Fédération Charles Hermite
		\item 5 départements et 28 équipes 
	\end{itemize}
\end{frame}

\subsection{ORPAILLEUR}
\begin{frame}
	\frametitle{ORPAILLEUR}
	\begin{itemize}
		\item Miguel Couceiro, Alian Gély, Amedeo Napoli
		\item Équipe ORPAILLEUR
		\item Agronomie, analyse de texte scientifiques et techniques, chimie organique et planification de synthèses, classification de signaux temporels, médecine et sidérurgie
		\item 4ème département : Traitement automatique des langues et des connaissances
	\end{itemize}
\end{frame}

\section{Définitions}

\subsection{Généralité}

\begin{frame}
	\frametitle{Formal Concept Analysis}
	\begin{columns}[] % align columns
		\begin{column}{.48\textwidth}
			\begin{center}
				$C(J, M, I)$\\
				\smallbreak
				$\uparrow \! 1 = \{1, 3\}$\\
				$\downarrow \! 3 = \{1, 3\}$\\
				\smallbreak
				 $3' = \{A\}$\\
				 $3'' = \{A\}' = \{1, 3\}$\\
				 \smallbreak
			\end{center}
		\end{column}%
		\hfill%
		\begin{column}{.48\textwidth}
			\only<1| handout:1> {
				\begin{figure}[H]
	\begin{center}
		\begin{tabular}{ l | c c c }
			 & A & B & C \\
			\hline
			1 & x &  & x \\
			2 &  & x & x \\
			3 & x &  &  \\
		\end{tabular}
		\caption{Contexte}
		\label{contexte_fca}
	\end{center}
				\end{figure}}
			\only<2| handout:2> {
				\begin{figure}[H]		
				\begin{center}
					\begin{tikzpicture}
						\node [bnode, label=below:{bot}] (bot) at (0,0) {};
						\node [bnode, label=left:{1}] (1) at (-1, 1) {};
						\node [bnode, label=right:{2, B}] (2) at (1, 1) {};
						\node [bnode, label=left:{3, A}] (3) at (-1, 2) {};
						\node [bnode, label=right:{C}] (4) at (1, 2) {};
						\node [bnode, label={top}] (top) at (0, 3) {};
			
						\path [line] (bot) -- (1);
						\path [line] (bot) -- (2);
						\path [line] (1) -- (3);
						\path [line] (1) -- (4);
						\path [line] (2) -- (4);
						\path [line] (3) -- (top);
						\path [line] (4) -- (top);
					\end{tikzpicture}
				\end{center}
				\caption{Treillis}
				\label{treillis_fca}
				\end{figure}}
		\end{column}%
	\end{columns}
\end{frame}

\subsection{Relations flèches}

\begin{frame}
	toto
\end{frame}
\subsection{Treillis disctributif}
\begin{frame}
	toto
\end{frame}

\section{Motivation}
\subsection{Ouverture}
\begin{frame}
	toto
\end{frame}

\section{Contribution}
\subsection{Implémentation}
\begin{frame}
	toto
\end{frame}
\subsection{Fusion}
\begin{frame}
	toto
\end{frame}
\subsection{Problèmes rencontrés}
\begin{frame}
	toto
\end{frame}
\subsection{Ouverture}
\begin{frame}
	toto
\end{frame}

\section{Bilan}
\subsection{Ouverture4}
\begin{frame}
	toto
\end{frame}
\subsection{Ouverture5}
\begin{frame}
	toto
\end{frame}

\begin{frame}
	\begin{columns}[T] % align columns
		\begin{column}{.48\textwidth}
			\begin{center}
				Merci de votre attention.\\
				Des question ?
			\end{center}
		\end{column}%
		\hfill%
		\begin{column}{.48\textwidth}
			\begin{itemize}
			 	\item ref1
			 	\item ref2
			 	\item ref3
			\end{itemize}
		\end{column}%
	\end{columns}
\end{frame}

\end{document}
