\chapter{Bilan}

\section{Contribution}

Nous cherchions à produire des treillis utilisables sous forme de graphes médians afin d'être utilisable dans la phylogénie en encodant l'ensemble des arbres parcimonieux dans un graphe unique. Pour se faire nous nous sommes basés sur les travaux de H.J. Bandelt, U. Priss et de l'équipe, en recherchant à générer un treillis dont les sous-treillis construits à partir des filtres de ses atomes sont distributifs et ceci à partir d'un treillis quelconque.

\bigbreak

Nous avons obtenu un programme permettant de le faire. Il commence par extraire le contexte initial d'un fichier pour ensuite regarder si tous les treillis générés à partir des filtres des atomes sont distributifs en calculant les relations flèches pour utiliser la propriété que le contexte réduit d'un treillis distributif possède uniquement une relation fléche double par ligne et par colonne et aucune autre relations fléches. Si ce n'est pas le cas, nous appliquons une méthode de transformation sur chacun d'eux avant de les remettre en commun dans un unique treillis. À partir de là, nous effectuons la fusion deux à deux de tous les couples de n\oe uds remplissant les conditions suivantes :
\begin{itemize}
	\item Les deux n\oe uds à fusionner ne doivent pas faire partis des n\oe uds du treillis de départ.
	\item Les deux n\oe uds à fusionner doivent être en couverture d'un n\oe ud présent dans au moins deux filtres d'atomes.
	\item Les deux n\oe uds à fusionner ne doivent pas avoir de n\oe uds en commun dans leurs idéaux une fois que les idéaux du n\oe ud en commun (celui dont il est question dans la condition précédente) leur sont retirés.
\end{itemize}
Une fois les fusions effectuées nous recommencons à l'étape de vérification de la distributivité des treillis des atomes. Une fois que nous l'atteignons le programme s'arrête et génére le treillis obtenu sous forme pdf et un fichier texte du contexte réduit exploitable par ConExp. Vous pourrez trouver en Annexe \ref{exemplescas} tous les cas qui ont été traité et testé pendant le développement. Tandis que l'Annexe \ref{inputoutput} présente le formatage du fichier source et du fichier d'exportage pour ConExp.

\section{Stage}

Ce stage a été bénéfique sur le plan technique pour les raisons classiques d'approfondissement des différents éléments dont je me suis servis tel que le langage Python, Latex et des notions vues en cours qui ont été appliqué sur un cas concret. Mais également sur le plan personnel, le Loria reste un lieu des plus intéressant pour échanger sur un sujet afin de faire avancer la problèmatique avec des personnes ayant des approches différentes pour s'approcher au mieux d'une solution fiable.

\bigbreak

Et surtout que la recherche est un domaine comprenant des personnes passionnées pas seulement par ce qu'elles font mais également par le simple fait d'avancer dans une parcelle d'inconnue pour faciliter un fragment de vie d'un inconnu. Elles en viennent à devoir faire des choix dont les conséquences sont encore inconnues et qu'elles doivent justifier au mieux à ce moment tout en devant les remettre en cause à chaque itération. Ce stage m'a permis de confirmer une nouvelle fois mon goût pour cette façon de faire, d'agir au mieux de nos connaissances à ce moment pour le simple besoin de vouloir faire les choses au mieux même si ce n'est pas pour notre propre profit direct.