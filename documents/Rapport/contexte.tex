\chapter{Contexte}

\section{Laboratoire}

Le Loria, Laboratoire lorrain de Recherche en Informatique et ses Applications est une Unité Mixte de Recherche (UMR 7503), commune à plusieurs établissements : le CNRS, l’Université de Lorraine et Inria.

\bigbreak

Le laboratoire a pour mission la recherche fondamentale et appliquée en sciences informatiques et ce, depuis sa création, en 1997.

\bigbreak

Il est membre de la Fédération Charles Hermite qui regroupe les trois principaux  laboratoires de recherche en mathématiques et STIC (Science et Technologies de l’Information et de la Communication) de Lorraine. Le laboratoire fait partie du pôle scientifique AM2I (Automatique, Mathématiques, Informatique et leurs interactions) de l’Université de Lorraine.

\bigbreak

Les travaux scientifiques sont menés au sein de 28 équipes structurées en 5 départements, dont 15 sont communes avec Inria, représentant un total de plus de 400 personnes. Le Loria est un des plus grands laboratoires de la région lorraine.

\section{Équipe ORPAILLEUR}

Le stage s'est déroulé au sein de l'équipe ORPAILLEUR et plus précisément en collaboration direct avec Miguel Couceiro, Alain Gély et Amedeo Napoli qui la dirige. Elle fait partie du 4\up{ème} département « Traitement automatique des langues et des connaissances ». 

\bigbreak

Le domaine de recherche de l'équipe est l'exploration de connaissances dans les bases de données (knowledge discovery in databases). Cela consiste à traiter des données afin d'en resortir des unités de connaissances utiles et réutilisables. Le processus se fait en trois mécanismes principaux : la préparation des données, leur exploitation et l'interprétation des unités extraites en unité de connaissance.

\bigbreak

Son nom vient de la comparaison que nous pouvons faire entre son activité et la recherche d'or. En effet en associant la présence d'or en tant qu'unités de connaissances et les données du terrain en tant que base de données, on obtient l'activité de l'équipe sur les données.